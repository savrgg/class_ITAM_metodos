
%Example of use of oxmathproblems latex class for problem sheets
%(un)comment this line to enable/disable output of any solutions in the file
%\printanswers
\documentclass{../oxmathproblems}
\usepackage{blindtext}
\usepackage{hyperref}
\usepackage{geometry}
%define the page header/title info
\course{ITAM - Métodos Estadísticos para C.Pol y R.I.}
\oxfordterm{Repaso de Estadística 2}
\sheetnumber{1}

\sheettitle{}

\extrawidth{2cm}

\begin{document}

\begin{questions}
\miquestion Un dado sin cargar se lanza tres veces. Sean $Y_1$, $Y_2$ y $Y_3$ el número de puntos vistos en la cara superior para los tiros 1,2 y 3, respectivamente. Suponga que estamos interesados en 
$ \bar{y}= \frac{(Y_1 + Y_2 + Y_3)}{3} $ 
el número promedio de puntos vistos en una muestra de tamaño 3. 

Determine: 
\begin{itemize}
\item a) Si el estimador es sesgado o insesgado
\end{itemize}

\miquestion Una máquina embotelladora puede ser regulada para que descargue un promedio de $\mu$ onzas por botella. Se ha observado que la cantidad de líquido dosificado por la máquina está distribuida normalmente con $\sigma$ = 1.0 onzas. Una muestra de $n = 9$ botellas seleccionadas de forma aleatoria de la producción de la máquina en un día determinado (todas embotelladas con el mismo ajuste de la máquina) y las onzas de contenido líquido se miden para cada una.

Determine 
\begin{itemize}
\item a) La probabilidad de que la media muestral se encuentre a no más de 0.3 onzas de la verdadera media $\mu$ para el ajuste seleccionado de la máquina. 
\item b) ¿Cuántas observaciones deben estar incluidas en la muestra si se desea que $\bar{y}$ se encuentre a no más de 0.3 onzas de $\mu$ con probabilidad 0.95?
\end{itemize}

\miquestion Las calificaciones de exámenes para todos los estudiantes de último año de preparatoria en cierto estado tiene en promedio 60 y una varianza de 64. Una muestra aleatoria de $n = 100$ estudiantes de una escuela preparatorio grande tuvo una calificación promedio de 58. ¿Hay evidencia suficiente para sugerir que el nivel de conocimiento de esta escuela sea inferior?


\miquestion Si se toman muestras aleatorias de forma iterada de tamaños $n_1 = 6$ y $n_2 =10$ de dos poblaciones normales con la misma varianza poblacional, encuentre el número $b$ tal que: 
$$ P(\frac{S_1^2}{S_2^2} \leq b) = 0.95 $$ 

\miquestion Los tiempos de servicio para los clientes que pasan por la caja en una tienda de venta al menudeo son variables aleatorias independientes con media de 1.5 minutos y varianza de 1.0. 
Calcule la  probabilidad de que 100 clientes puedan ser atendidos en menos de 2 horas de tiempo total de servicio. 

\miquestion En las elecciones para la gobernatura de la CDMX, la candidata A piensa que puede ganar las elecciones de la ciudad si obtiene por lo menos el 55$\%$ de los votos de la CDMX. También piensa que alrededor del 50 por ciento de los votantes van a estar a su favor. Si se toma una muestra de 100 votantes se presentan a votar. ¿Cuál es la probabilidad de que la candidata A reciba al menos el 55$\%$ de sus votos?










\textbf{Bibliografía}
Wackerly. (2008). Estadística Matemática con Aplicaciones (7.a ed.). Cengage Learning.


\end{questions}
\end{document}






















