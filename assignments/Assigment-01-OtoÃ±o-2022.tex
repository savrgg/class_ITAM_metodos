

\documentclass{../oxmathproblems}
\usepackage{blindtext}
\usepackage{hyperref}
\usepackage{geometry}
\usepackage{geometry}

\course{ITAM - Métodos de Pronósticos}
\oxfordterm{Assignment 01}
\sheetnumber{1}
\sheettitle{}


\begin{document}

\begin{questions}

\miquestion Un contratista estima que sus tiempos de terminación para un proyecto, con sus respectivas probabilidades, son: 

\begin{center}
\begin{tabular}{ |c|c| } 
 \hline
 \textbf{Tiempo de terminación (t)} & \textbf{Probabilidad} \\ 
 \hline
 10 días  & 0.3 \\
 15 días  & 0.2 \\
 22 días & 0.5\\ 
 \hline
\end{tabular}
\end{center}


Calcular: 

\begin{itemize}
\item a) ¿Cuál es el número de días esperado para la terminación del proyecto?¿Cuál es su varianza?
\item b) Si la utilidad U que recibe el contratista depende del tiempo de terminación del proyecto, ¿cuál es el valor esperado y la desviación estándar de la utilidad si ésta última está dada por: 
\begin{enumerate}
\item  $U(t) = 7503 - 150t $
\item $U(t) = 3t^2 + 7503 $ 

\end{enumerate}
\end{itemize}

\miquestion Considere la siguiente función de probabilidad: 

\begin{center}
\begin{tabular}{ |c|c| } 
 \hline
 \textbf{x} & \textbf{f(x)} \\ 
 \hline
 2  & 0.1 \\
6  & 0.2 \\
 10 & 0.3\\ 
  14 & 0.3\\ 
   120 & 0.1\\ 
 \hline
\end{tabular}
\end{center}


Calcular: 

\begin{itemize}
\item a) $E(x)$ y $Var(x)$
\item b) $E(x+5)$ y $Var(x+5)$
\item c) $E(x^2)$
\item d) $E[(x-6)^2]$
\end{itemize}


\miquestion Las lecturas de temperatura de un termopar puesto en un medio a temperatura constante se distribuye normal con media $\mu$, la temperatura verdadera de la mediua y, desviación estándar, $\sigma$. ¿Cuál debe ser el valor de  $\sigma$ que asegure que el 95 por ciento de las lecturas están entre 0.1 grados $\mu$? 


\miquestion Un banco reportó al Gobierno Federal que sus cuentas de ahorro tienen un saldo promedio de 1890 y una desviación estándar de 264. Los auditores del Gobierno Federal selecciona aleatoriamente 144 cajas de ahorro para comprobar la confiabilidad del reporte dado por el banco. Éstos van a certificar el reporte del banco solo si el saldo medio de la muestra difiere a lo más en 50 del saldo medio reportado. ¿Cuál es la probabilidad de que los auditores no certifiquen el informe del banco?



\miquestion De una población con media $\mu$ y varianza $\sigma^2$ se extraen dos muestras aleatorias simples e independientes de tamaños $n_1$ y $n_2$ = $n_1/2$. Sus medias muestrales son $\bar{x}_1$ y $\bar{x}_2$ respectivamente. Para estimar $\mu$ se proponen tres estimadores: 


$ \mu_1 = \bar{x}_1$

$ \mu_2 = \bar{x}_2$

$ \mu_3 =  \frac{\bar{x}_1 + \bar{x}_2}{2} $ 

\begin{itemize}
\item a) Indique si son insesgados 
\item b) Encuentre su varianza
\item c) ¿Cuál de los tres estimadores es mejor?
\end{itemize}


\miquestion Los costos variables, principalmente la mano de obra, hacen que los costos de la construcción de las casas varíen de una edificación a otra. Un constructor de casas necesita una ganancia media por encima de 8500 por casa para alcanzar una gananzia anual establecida como meta. Las ganancias por casa para las cinco edificaciones más recientes del constructor, medidas en dólares son: {8760,6370,9620,8200,10350} respectivamente.

\begin{itemize}
\item a) Encuentre un intervalo de confianza al 95 $\%$  para el promedio de la ganancia del constructor. Indique los supuestos sobre los cuales basa su desarrollo. 
\item b) Con la información obtenida en el inciso anterior, ¿sería razonable pensar que el constructor está trabajando al nivel de la ganancia deseada?
\end{itemize}

\miquestion  En una muestra aleatoria de 30 compañías de manufacturas, con activos fijos por debajo de $\$$ 10 000, se obtuvieron utilidades promedio del 1.18 $\%$ con una desviación estándar de 0.4 $\%$. 
En otra muestra seleccionada al azar de 40 compañías de manufacturas con activos fijos entre $\$$ 10 000 y $\$$ 50 000, las utilidades promedio y su desviación estándar fueron de 2.4 $\%$ y 0.6 $\%$ respectivamente. 

Con una significancia de 0.05, ¿se puede afirmar que la diferencia de las utilidades medias es igual a cero? ¿Qué se requiere suponer sobre el comportamiento de dichas variables?


\end{questions}

\end{document}









