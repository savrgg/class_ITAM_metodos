
\documentclass{../oxmathproblems}
\usepackage{blindtext}
\usepackage{hyperref}
\usepackage{geometry}
%define the page header/title info
\course{ITAM - Métodos Estadísticos para C.Pol y R.I.}
\oxfordterm{Assignment 03 }
\sheetnumber{1}

\sheettitle{}

\extrawidth{3cm}

\begin{document}
\begin{questions}

\miquestion Es frecuente que a los auditores se les exija comparar el valor auditado (o de lista) de un artículo de inventario contra el valor en libros. Si una empresa está llevando su inventario y libros actualizados, debería haber una fuerte relación lineal entre dichos valores. Una empresa muestreó diez artículos de inventario y obtuvo los valores auditado y en libros que se dan en la tabla siguiente. 

\begin{tabular}{|c|c|c|}
\hline
Artículo & Valor en libros $(x_i)$ & Valor auditado$(y_i)$ \\ \hline
1 & 9 & 10\\
2 & 14 & 12\\
3 & 7 & 9\\ 
4 & 29  & 27\\ 
6 & 45 & 47\\ 
7 & 109 & 112\\ 
8 & 40 & 36\\ 
9 & 238 & 241\\ 
10 & 60 & 59\\
11 & 170 & 167\\
12 & 1 & 250\\
13 & 30 & 780\\
\hline
\end{tabular}


\textbf {Sección 1} 

\begin{enumerate}
  \item Mediante el método de mínimos cuadrados encuentre los estimadores de los paramétros $\beta_0$ y $\beta_1$. (Enuncie los supuestos que aplican). 
  \item Determine el valor de TSS, ESS, RSS  $R^2$ ,$R^2$ ajustada   e interprete 
  \item Determine las varianzas de los estimadores ($\hat{\beta_0}$ y $\hat{\beta_1}$) y la covarianza  ($\hat{\beta_0}$ y $\hat{\beta_1}$), determine $\hat{\sigma^2}$ y $\hat{\sigma}$ y qué concluye a partir de ello.
  \item Realice una prueba de hipótesis para la correlación entre el valor en libros $(x_i)$ y el valor auditado $(y_i)$
  \item Realice una prueba de hipótesis para determinar si la  $\beta_0$ y $\beta_1$ son significativas (solo con valor p) 
  \item Intervalo de confianza al 95\% para $\beta_0$ y $\beta_1$ 
   \item Prueba F 
   \item Predicción media cuando x= 100. Intervalo de confianza para la predicción media cuando $x* = 100$
   \item Intervalo de confianza para la predicción individual cuando x* = 100
   \item Muestra el valor medio de los residuos es 0 ( $\sum_{i}\hat{\epsilon_i}=0$ ), x es ortogonal a residuo, $\hat{y}$  es ortogonal al residuo prueba 
\end{enumerate}




\textbf {Sección 2} 
\text{ Dato atípico: } 
\begin{enumerate}
  \item Elimina el dato atípico y vuelve a cálcular las betas ($\beta_0$ y $\beta_1$)
\end{enumerate}

\textbf {Sección 3} 
\begin{enumerate}
  \item Aplica la prueba Jarque Bera para normalidad 
\end{enumerate}



\textbf{Bibliografía}
Wackerly. (2008). Estadística Matemática con Aplicaciones (7.a ed.). Cengage Learning.



\end{questions}
\end{document}

