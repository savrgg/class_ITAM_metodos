\documentclass{../oxmathproblems}
\usepackage{blindtext}
\usepackage{hyperref}
\usepackage{geometry}
%define the page header/title info
\course{ITAM - Métodos Estadísticos para C.Pol y R.I.}
\oxfordterm{Assignment 02 - Respuestas  }
\sheetnumber{1}

\sheettitle{}

\extrawidth{3cm}

\begin{document}
\begin{questions}


\miquestion 
\begin{itemize}
\item 
Sabemos que: 

\text {Los estimadores de mínimos cuadrados de $\alpha$ y $\beta$ son}

$$ b = \frac{S_{xy}}{S_{xx}} $$  \text {y} 
$$a = \bar{y} - b\bar{x}$$

\text {Entonces, para encontrar la recta de predicción de mínimos cuadrados para los datos de la tabla: } 

\textbf {Paso 1: }

$ S_{xx} = \sum{x_i^2} - \frac{\sum{x_i}^2}{n} = 23 634 - \frac{460^2}{10} = 2474$

$ S_{xy} = \sum{x_iy_i} - \frac{\sum{x_i}\sum{y_i}}{n} $

$ = 36 854 - \frac{460*760}{10} = 1894 $ 


$ \bar{y} = \frac{\sum{y_i}}{n} = \frac{760}{10} = 76 $ 

$  \bar{x} = \frac{\sum{x_i}}{n} = \frac{460}{10} = 46$ 


\textbf {Paso 2: }

$ b = \frac{S_{xy}}{S_{xx}}  = \frac{1894}{2474} = 0.76556 $

\text{y} 

$$a = \bar{y} - b\bar{x} = 76 - (0.76556)(46) = 40.78424 $$

\text{Entonces, la recta de regresión de mínimos cuadrados es:} 


$$ \hat{y} = a + bx $$  

 \text{o bien:} 
 
$$ \hat{y} = \beta_0 + \beta_1x = 40.78424 + 0.76556x$$ 

\item Para determinar si hay una relación lineal significativa entre las calificaciones realizamos una prueba de hipótesis: 

\text{La hipótesis a probar son}

$H0: \beta = 0$ \text{contra}  $Ha: \beta  \neq 0$

\text{ y el valor observado del estadístic de prueba se calcula como: }


$ t = \frac{b- 0}{\sqrt MSE/S_xx} = \frac{0.7656 - 0}{\sqrt 75.7532/2474} = 4.38 $ 

\text{con}  $(n-2) = 8$  \text{grados de libertad. Con } $ \alpha = 0.05$, \text{se puede rechazar Ho cuando} $t > 2.306$  o $t < -2.306$ 
\text{Como el valor observado (el estadístico de prueba) cae en la región de rechazo, Ho es rechazada y se puede concluir}

\text{que "hay una relación lineal significativa entre las calificaciones esperadas y la puntuación final del examen.}

\item Para estimar el promedio de las calificaciones cuyo aprovechamiento es de 50, con un intervalo de confianza debemos: 

\text{La estimación puntual de}  $ E(y\mid x_0 = 50) $ \text{el promedio de calificación es: }

$ \hat{y} = 40.78424 + 0.76556(50) = 79.06 $ 
\end{itemize} 

\miquestion 
\begin{itemize}
\item 

\text{Sabemos que: }

$ S_{xx} = \sum{x_i^2} - \frac{\sum{x_i}^2}{n}$  

$ S_{xy} = \sum{x_iy_i} - \frac{\sum{x_i}\sum{y_i}}{n} $

$ S_{yy} = \sum{y_i^2} - \frac{\sum{y_i}^2}{n}$  

\text{Entonces: }
$S_{xx} = 60.4$ 
$S_{xy} = 328 $ 

$S_{yy} = 2610 $

$$ r = \frac{S_{xy}}{ \sqrt{S_{xx}S_{yy}}} $$ 

\text{Entonces: }

$r = \frac{328}{\sqrt{60.4*2610}} = 0.8261 $ 

\text{El valor obtenido para r es cerca a 1, indica que hay una relación lineal positiva bastante fuerte entre estatura  y peso.}

\item 
\text{Para probar si es significativamente diferente de cero, tenemos: }

$ Ho: $



\end{itemize}




\textbf{Bibliografía}
Wackerly. (2008). Estadística Matemática con Aplicaciones (7.a ed.). Cengage Learning.


\end{questions}
\end{document}



