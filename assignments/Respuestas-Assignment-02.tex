\documentclass{../oxmathproblems}
\usepackage{blindtext}
\usepackage{hyperref}
\usepackage{geometry}
%define the page header/title info
\course{ITAM - Métodos Estadísticos para C.Pol y R.I.}
\oxfordterm{Assignment 02 - Respuestas  }
\sheetnumber{1}

\sheettitle{}

\extrawidth{3cm}

\begin{document}
\begin{questions}


\miquestion 
\begin{itemize}
\item 
Sabemos que: 

\text {Los estimadores de mínimos cuadrados de $\alpha$ y $\beta$ son}

$$ b = \frac{S_{xy}}{S_{xx}} $$  \text {y} 
$$a = \bar{y} - b\bar{x}$$

\text {Entonces, para encontrar la recta de predicción de mínimos cuadrados para los datos de la tabla: } 

\textbf {Paso 1: }

$ S_{xx} = \sum{x_i^2} - \frac{\sum{x_i}^2}{n} = 23 634 - \frac{460^2}{10} = 2474$

$ S_{xy} = \sum{x_iy_i} - \frac{\sum{x_i}\sum{y_i}}{n} $

$ = 36 854 - \frac{460*760}{10} = 1894 $ 


$ \bar{y} = \frac{\sum{y_i}}{n} = \frac{760}{10} = 76 $ 

$  \bar{x} = \frac{\sum{x_i}}{n} = \frac{460}{10} = 46$ 


\textbf {Paso 2: }

$ b = \frac{S_{xy}}{S_{xx}}  = \frac{1894}{2474} = 0.76556 $

\text{y} 

$$a = \bar{y} - b\bar{x} = 76 - (0.76556)(46) = 40.78424 $$

\text{Entonces, la recta de regresión de mínimos cuadrados es:} 


$$ \hat{y} = a + bx $$  

 \text{o bien:} 
 
$$ \hat{y} = \beta_0 + \beta_1x = 40.78424 + 0.76556x$$ 
\item %Calcular el coeficiente de correlación 


\item %HACER LA PRUEBA DE HIPÓTESIS PARA EL COEFICIENTE DE CORRELACIÓN 

\text{que "hay una relación lineal significativa entre las calificaciones esperadas y la puntuación final del examen.}

\item Para estimar el promedio de las calificaciones cuyo aprovechamiento es de 50, con un intervalo de confianza debemos: 

\text{La estimación puntual de}  $ E(y\mid x_0 = 50) $ \text{el promedio de calificación es: }

$ \hat{y} = 40.78424 + 0.76556(50) = 79.06 $ 
\end{itemize} 

\miquestion 
\begin{itemize}
\item 

\text{Sabemos que: }

$ S_{xx} = \sum{x_i^2} - \frac{\sum{x_i}^2}{n}$  

$ S_{xy} = \sum{x_iy_i} - \frac{\sum{x_i}\sum{y_i}}{n} $

$ S_{yy} = \sum{y_i^2} - \frac{\sum{y_i}^2}{n}$  

\text{Entonces: }
$S_{xx} = 60.4$ 
$S_{xy} = 328 $ 

$S_{yy} = 2610 $

$$ r = \frac{S_{xy}}{ \sqrt{S_{xx}S_{yy}}} $$ 

\text{Entonces: }

$r = \frac{328}{\sqrt{60.4*2610}} = 0.8261 $ 

\text{El valor obtenido para r es cerca a 1, indica que hay una relación lineal positiva bastante fuerte entre estatura  y peso.}

\item 
\text{Para probar si es significativamente diferente de cero, tenemos: }

$ H_o: \rho = 0 $  \text{contra}  $ H_a : \rho \neq 0 $ 

\text{ El valor del estadístico de prueba es} 

$ t = r\sqrt \frac{n-2}{1-r^2}$ 

$ = 0.8261*\sqrt(\frac{10-2}{1-(0.8261)^2}) = 4.15  $ 

\text{La $t$ observada o el valor de tablas: tiene una distribución, con $ n = 10$,  de 8 grados de libertad. }

\text{ Dado que la $t$ observada es mayor que}

$ t_0.005 = 3.355 $ 

\text{y el $valor p$ es menor a $2(0.005) = 0.01$, entonces rechazamos Ho. Concluimos que la correlación es 
}

\text{significativamente diferente de 0}

\end{itemize}

\miquestion 
\begin{itemize}
\item 

\text{Tenemos  que: }

$$ \hat{\beta_1} = \frac{S_xy}{S_xx} $$ 
$$ \hat{\beta_0} = \bar{y} -\hat{\beta_1}\bar{x}  $$

\text{Entonces: }


$ S_{xx} = \sum{x_i^2} - \frac{\sum{x_i}^2}{n}$  

$ S_{xy} = \sum{x_iy_i} - \frac{\sum{x_i}\sum{y_i}}{n} $

$ S_{yy} = \sum{y_i^2} - \frac{\sum{y_i}^2}{n}$  

$$ \hat{\beta_1} = \frac{7- \frac{1}{5}(0)(5)}{10- \frac{1}{5}(0)^2} = 0.7  
$$ 

\text{Ahora: }

$ \bar{y} = \frac{\sum(y_i)}{n} = 0.7 $

$ \bar{x} = \frac{\sum(x_i)}{n} =  0 $ 

$$
\hat{\beta_0} = \frac{5}{5} - (0.7)(0) = 1 
$$ 
\text{Entonces, la recta de regresión de mínimos cuadrados es:} 


$$ \hat{y} = \bar{y} +\hat{\beta_0}x $$ 

$$  \hat{y} = 1 + 0.7x $$ 

\item Ahora, para determinar el intervalo de confianza al 90$\%$ para E(y) cuando x= 1 tenemos que: 

$ E(y) = \beta_0 + \beta_1x$ 

\text{Para estimar dicho valor fijo, usaremos el estimador insesgado } 
$ \hat{E(Y)} = \hat{\beta_0} + \hat{\beta_1}x* $ \text{Entonces:} 

$ \hat{E(Y)} = 1+ 0.7x* $ 
\text{En este caso, } $ x* = 1$ \text{y como $n= 5$,} $\bar{x} = 0 $ y $S_xx = 10 $ 


\text{Entonces, el error estándar es}

$ \frac{1}{n} + \frac{(x-\bar{x})^2}{S_xx} = \frac{1}{5} + \frac{(1-0)^2}{10} = 0.3 $

\text{Ahora bien, el valor de la $t_0.05$ de tablas es con n-2= 3 graods de libertad es 2.353}


\text{El intervalo está dado por: }
$$ 
\hat{\beta_0} + \hat{\beta_1}x* \pm t_{\alpha/2} *S* \sqrt {\frac{1}{n} + \frac{(x-\bar{x})^2}{S_xx}}
$$ 


\end{itemize}


\textbf{Bibliografía}
Wackerly. (2008). Estadística Matemática con Aplicaciones (7.a ed.). Cengage Learning.


\end{questions}
\end{document}



