\documentclass{../oxmathproblems}
\usepackage{blindtext}
\usepackage{hyperref}
\usepackage{geometry}
%define the page header/title info
\course{ITAM - Métodos Estadísticos para C.Pol y R.I.}
\oxfordterm{Assignment 02 - Respuestas  }
\sheetnumber{1}

\sheettitle{}

\extrawidth{3cm}

\begin{document}
\begin{questions}


\miquestion Se realizó una entrevista a 10 estudiantes acerca de sus calificaciones  esperadas y sus puntos finales en el primer parcial de matemáticas. 

\begin{tabular}{|c|c|c|c|c|c|c|}
\hline
Estudiante ($i$) & Calificación esperada ($X_i$) & Puntos finales ($Y_i$) & $\hat{Y}$ & $(Y_i - \hat{Y_i})^2$ & $(Y_i - \bar{Y_i})^2$ & $(\hat{Y_i}-\bar{Y_i})^2$\\ \hline
1 & 39 & 65 & 70.6 & 31.8216 & 121 & 28.7182\\
2 & 43 & 78 & 73.7 & 18.4615 & 4 & 5.2748\\
3 & 21  & 52 & 56.9 & 23.6289 & 576 & 366.3031\\ 
4& 64 & 82 & 89.8 & 60.5302 & 36 & 189.8915\\ 
5 & 57 & 92 & 84.4 &  57.4385 & 256 & 70.9163\\ 
6 & 47 & 89 & 76.8 & 149.6815 & 169 & 0.5861\\ 
7 & 28 & 73 & 62.2 & 116.2108 & 9 & 189.8915\\
8 & 75 & 98 & 98.2 & 0.0405 & 484 & 492.8974\\ 
9 & 34 & 56 & 66.8 & 116.9265 & 400 & 84.3962\\ 
10 & 52 & 75 & 80.6 & 31.2858 & 1 & 21.0991\\ 
\hline
\end{tabular}

\begin{itemize}
\item Encuentre la recta de predicción de mínimos cuadrados para los datos de las calificaciones del primer parcial de matemáticas.

Se tiene que $\bar{x} = 46$, $ \bar{y} = 76$, $S_{x}^2 = 274.89$, $S_{y}^2 = 228.44$,  $S_{xy}=210.44$ $n = 10$,

$$ \beta_1 = \frac{S_{xy}}{S_{xx}} = 0.7656$$
$$ \beta_0 = \bar{y} - \beta_0 \bar{x} = 40.7841$$
$$ \hat{y} = \beta_0+\beta_1 x = 40.7841 + 0.7656x$$

\item Determine el coeficiente de correlación

$$ r_{xy} = \frac{S_{xy}}{S_{x} S_{y}} =  0.8398 $$

\item ¿La correlación es significativamente distinta de cero? 
$$H_0: \rho_{xy} = 0 \qquad vs \qquad H_1 \neq 0$$
$$T_{S_{xy}} = \frac{r_{xy}\sqrt{n-2}}{\sqrt{1-r_{xy}^2}} = 4.3750$$
$$valorp = 0.0024$$

\newpage

\item Determine si hay una relación lineal significativa entre las calificaciones esperadas y los puntos finales. Nota: realize una prueba de hipótesis para $\beta_0$ y $\beta_1$

Necesitamos calcular:
$$ RSS = 606.03 \qquad ESS = 1449.98 \qquad TSS = 2056$$
$$ \hat{\sigma^2} = \frac{606.03}{8} = 75.7532$$
$$ \hat{\sigma_{\beta_0}^2}= \hat{\sigma^2}\left(\frac{1}{n}+\frac{\bar{x}^2}{S_x^2 (n-1)}\right)  = 72.3664$$
$$ \hat{\sigma_{\beta_1}^2}= \frac{\hat{\sigma^2}}{S_x^2 (n-1)} = 0.03062$$

Para $\beta_0$:
$$ H_0:\beta_0 = 0  \qquad \text{vs} \qquad H_1:\beta_0 \neq 0 $$
$$ T_{\beta_0} = \frac{\hat{\beta_{0}}-\beta_0}{\sqrt{\sigma^2_{\beta_0}}} = 4.7943$$ 
$$ valorp = 0.0014 $$ 
Para $\beta_1$:
$$ H_0:\beta_1 = 0  \qquad \text{vs} \qquad H_1:\beta_1 \neq 0 $$
$$ T_{\beta_1} = \frac{\hat{\beta_{1}}-\beta_1}{\sqrt{\sigma^2_{\beta_1}}} = 4.375015$$ 
$$ valorp = 0.0024 $$

\item Estime el promedio de las calificaciones para estudiantes con una puntuación de aprovechamiento es 50, con un intervalo de confianza de 95%

$$\hat{y_{50}} = \hat{\beta_0}+\hat{\beta_1}(50) = 79.0623$$
$$ t_{.025, (8)} = 2.3060$$
$$\left(\hat{y_{50}} \pm t_{.025, (8)}*\sqrt{\hat{\sigma^2}\left(\frac{1}{n}+\frac{(50-\bar{x})^2}{S_x^2 (n-1)}\right)}\right) = (72.51334, 85.61115)$$

\end{itemize}

\miquestion 
\begin{itemize}
\item 

\text{Sabemos que: }

$ S_{xx} = \sum{x_i^2} - \frac{\sum{x_i}^2}{n}$  

$ S_{xy} = \sum{x_iy_i} - \frac{\sum{x_i}\sum{y_i}}{n} $

$ S_{yy} = \sum{y_i^2} - \frac{\sum{y_i}^2}{n}$  

\text{Entonces: }
$S_{xx} = 60.4$ 
$S_{xy} = 328 $ 

$S_{yy} = 2610 $

$$ r = \frac{S_{xy}}{ \sqrt{S_{xx}S_{yy}}}   $$ 

\text{Entonces: }

$r = \frac{328}{\sqrt{60.4*2610}} = 0.8261 $ 

\text{El valor obtenido para r es cerca a 1, indica que hay una relación lineal positiva bastante fuerte entre estatura  y peso.}

\item 
\text{Para probar si es significativamente diferente de cero, tenemos: }

$ H_o: \rho = 0 $  \text{contra}  $ H_a : \rho \neq 0 $ 

\text{ El valor del estadístico de prueba es} 

$ t = r\sqrt \frac{n-2}{1-r^2}$ 

$ = 0.8261*\sqrt(\frac{10-2}{1-(0.8261)^2}) = 4.15  $ 

\text{La $t$ observada o el valor de tablas: tiene una distribución, con $ n = 10$,  de 8 grados de libertad. }

\text{ Dado que la $t$ observada es mayor que}

$ t_0.005 = 3.355 $ 

\text{y el $valor p$ es menor a $2(0.005) = 0.01$, entonces rechazamos Ho. Concluimos que la correlación es 
}

\text{significativamente diferente de 0}

\end{itemize}

\miquestion 
\begin{itemize}
\item 

\text{Tenemos  que: }

$$ \hat{\beta_1} = \frac{S_xy}{S_xx} $$ 
$$ \hat{\beta_0} = \bar{y} -\hat{\beta_1}\bar{x}  $$

\text{Entonces: }


$ S_{xx} = \sum{x_i^2} - \frac{\sum{x_i}^2}{n}$  

$ S_{xy} = \sum{x_iy_i} - \frac{\sum{x_i}\sum{y_i}}{n} $

$ S_{yy} = \sum{y_i^2} - \frac{\sum{y_i}^2}{n}$  

$$ \hat{\beta_1} = \frac{7- \frac{1}{5}(0)(5)}{10- \frac{1}{5}(0)^2} = 0.7  
$$ 

\text{Ahora: }

$ \bar{y} = \frac{\sum(y_i)}{n} = 0.7 $

$ \bar{x} = \frac{\sum(x_i)}{n} =  0 $ 

$$
\hat{\beta_0} = \frac{5}{5} - (0.7)(0) = 1 
$$ 
\text{Entonces, la recta de regresión de mínimos cuadrados es:} 


$$ \hat{y} = \bar{y} +\hat{\beta_0}x $$ 

$$  \hat{y} = 1 + 0.7x $$ 

\item Ahora, para determinar el intervalo de confianza al 90$\%$ para E(y) cuando x= 1 tenemos que: 

$ E(y) = \beta_0 + \beta_1x$ 

\text{Para estimar dicho valor fijo, usaremos el estimador insesgado } 
$ \hat{E(Y)} = \hat{\beta_0} + \hat{\beta_1}x* $ \text{Entonces:} 

$ \hat{E(Y)} = 1+ 0.7x* $ 
\text{En este caso, } $ x* = 1$ \text{y como $n= 5$,} $\bar{x} = 0 $ y $S_xx = 10 $ 


\text{Entonces, el error estándar es}

$ \frac{1}{n} + \frac{(x-\bar{x})^2}{S_xx} = \frac{1}{5} + \frac{(1-0)^2}{10} = 0.3 $

\text{Ahora bien, el valor de la $t_0.05$ de tablas es con n-2= 3 grados de libertad es 2.353}


\text{El intervalo está dado por: }
$$ 
\hat{\beta_0} + \hat{\beta_1}x* \pm t_{\alpha/2} *S* \sqrt {\frac{1}{n} + \frac{(x-\bar{x})^2}{S_xx}}
$$ 

\text{sustituyendo:  }
$$
((1+(0.7)(1)) \pm (2.353)(0.606)\sqrt{0.3}
$$
$
= 1.7 \pm 0.781 
$ 
\text{El intervalo está dado por:  }
$$ (0.919,2.481) $$ 
\text{Es decir:   }
\text{Con un nivel de confianza de 90\%  cuando la variable independiente tome el valor de 1, la variable dependiente variará entre 0.919 y 2.481}

\item Para demostrar que la pendiente difiere de cero dicho ejercicio implica una prueba de hipótesis


$ H_o: \beta_1 = 0 $  \text{contra}  $ H_a : \beta_1 \neq 0 $ 

\text{ El valor del estadístico de prueba es} 

$ t = \frac{\hat{\beta_1}}{s\sqrt{c_{11}}}$ 

\text{ Sustituyendo tenemos:}

$$ c_{11} = \frac{1}{S_xx} = 1/10 = 0.1 $$ 

$ t = \frac{0.7-0}{0.606\sqrt{0.1}}  = 3.65 $ 

\text{ Ahora, si tomamos con un $\alpha = 0.05$, el valor de t de tablas , $t_{\alpha/2} = t_{0.025} $ con 3 grados de libertad es $3.182$ } 

\text{Bajo la hipótesis alternativa la región de rechazo está dado por: }
$$ -3.182 \leq t \geq 3.182  $$ 

\text{Ahora bien, dado que es una prieba de dos colas para calcular el valor p: }

$$ valor p = 2P(t > 3.65) $$ 

$$ = 0.01 < P(t > 3.65) < 0.025 $$ 
\text{Entonces: } 
$$ = 0.02 < valor p < 0.05 $$ 

\text{Dado que el p value es menor a 0.1, rechazamos la hipótesis nula } 

\item  Para calcular el intervalo de confianza para $\beta_1$ tenemos: 

\text { El valor de tablas para $t_{0.025} $ con 3 grados de libertad es 3.182. Entonces el intervalo de confianza está dado por} 

$$ \hat{\beta_1} \pm t_{0.025} *s* \sqrt {c_{11}} $$ 
\text {siendo $c_{11}  = 1/ S_{xx} $}

\text{Susituyendo, obtenemos } 

$$ 0.7 \pm (3.182)(0.606)\sqrt{0.1} $$ 

\text{O bien, } 

$$ 0.7 \pm 0.610 $$

\text{El parámetro de $\beta_1$ variará entre 0.9 y 1.31 unidades } 


\item Para determinar el pronóstico puntual cuando x = 2 con un nivel de confianza de 0.90. Tenemos:

$$ \hat{\beta_0} = 1$$
$$ \hat{\beta_1} = 0.7 $$ 

\text{ De modo que el valor pronósticado para Y cuando $ x = 2 $ es } 
 $$ \hat{\beta_0} + \hat{\beta_1}x* = 1 + (0.7)(2) = 2.4 $$ 
 
 
\text{ Además,  el intervalo de predicción está dado por: } 

$$ \hat{\beta_0} +\hat{\beta_1}x* \pm t_{\alpha/2} S \sqrt{1+\frac{1}{n} + \frac{(x*-\bar{x})^2}{S_{xx}}}$$ 

\text{ Sustituyendo, tenemos: } 

$$ 2.4 \pm (2.353)(0.606)\sqrt{1+0.6} $$ 

$$ 2.4 \pm 1.804 $$ 
\end{itemize}

\miquestion 
\begin{itemize}
\item Para ajustar el modelo de regresión de mínimos cuadrados: 

$ S_{xx} = \sum{x_i^2} - \frac{\sum{x_i}^2}{n}$  

$ S_{xy} = \sum{x_iy_i} - \frac{\sum{x_i}\sum{y_i}}{n} $

$ S_{yy} = \sum{y_i^2} - \frac{\sum{y_i}^2}{n}$  

$$ \hat{\beta_1} = \frac{S_xy}{S_xx} $$ 
$$ \hat{\beta_0} = \bar{y} -\hat{\beta_1}\bar{x}  $$

\text{Sustituyendo, obtenemos: } 

$ S_{xy} = 8.709 - \frac{1}{6}(8.74)(6.148) = - 0.247 $ 

$ S_{xx} = 12.965 - \frac{1}{6}(8.74)^2 = 0.234 $ 

$ S_{yy} = 6.569 - \frac{1}{6}(6.148)^2 = 0.269 $ 

$$ \hat{\beta_1} = \frac{- 0.247}{0.234} = -1.056  $$ 
$$ \hat{\beta_0} =\frac{6.148}{6} - (-1.056)(\frac{8.74}{6}) = 2.563 $$ 

\text{Entonces, la recta de minimos cuadrados está dada por }

$$ \hat{\beta_0} + \hat{\beta_1}x =  -1.056 + 2.563x $$  

\item Prueba de hipótesis para $\beta_1$ 

\item Para realizar la prueba de hipótesis necesitamos: 
\begin{itemize}
\item Para $\beta_1$ 


$ H_o:\beta_1 = 0 $ \text{contra: } $ H_a:\beta_1 \neq 0 $ 


\text{ Bajo Ho el estadístico de prueba es: } 

$ T_{\beta_1}= \frac{(\beta{_1}-0)}{\sqrt(\sigma^2_{\beta_1})} =  -9.8263 $ 

\text{ El valor p : (en R) } 

$ 2*(1-pt(T_{\beta_1}, df = 5-2)) $ 

\text{ se obtiene : } 

$ valor p = 1.9977 $ 

\end{itemize}

\item Para calcular intervalo de confianza para $\beta_1$ al $90\%$

\text{Necesitamos: } 

$ \beta_1  \epsilon  ( \hat{\beta_1} \pm t_{\alpha^2,n-2} * \sqrt{Var(\hat{\beta_1})} $  

\text{Sustituyendo:  } 

$ t_{-10/2,6-2} = 0.07510391$ 
\text{cálculo de T  en R} 

$pt(0.05,df = 4, lower.tail= false)$ 

\text{El intervalo está dado por:  } 


$ \beta_1  \epsilon (-1.0655 \pm 0.07510391 * \sqrt{0.1084} ) $ 

\end{itemize}


\textbf{Bibliografía}
Wackerly. (2008). Estadística Matemática con Aplicaciones (7.a ed.). Cengage Learning.


\end{questions}
\end{document}



