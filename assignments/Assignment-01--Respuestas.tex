

\documentclass{../oxmathproblems}
\usepackage{blindtext}
\usepackage{hyperref}
\usepackage{geometry}
%define the page header/title info
\course{ITAM - Métodos Estadísticos para C.Pol y R.I.}
\oxfordterm{Respuestas: Repaso de Estadística 2}
\sheetnumber{1}

\sheettitle{}

\extrawidth{3cm}

\begin{document}
\begin{questions}



\miquestion 

\text{Sabemos que }
\text{para determinar el sesgo}

$$ Bias (\hat{\mu_1}) = E(\hat{\mu_1}) - \mu_1 $$ 

\text{Entonces: }

$$ E(\hat{\mu_1}) = E(\bar{y}) = E(\frac{y_1+y_2+y_3}{3})  $$ 

$$ = \frac{1}{3} ((E(y_1+y_2+y_3)) $$ 

\text{Se puede resolver de dos formas: primero }

$$
= \frac{1}{3} * 3E(y_i) = E(Y_i) = \mu_i
$$ 

\text{o bien: }

$$ \frac{1}{3} ((E(y_1+y_2+y_3)) = \frac{1}{3} (\mu_1 + \mu_2 + \mu_3) = \frac{1}{3}* 3 \mu_i = \mu_i $$ 

\text{Entonces: }

$$ Bias (\hat{\mu_1}) = E(\hat{\mu_1}) - \mu_1 = \mu_i - \mu_i = 0 $$ 

\text{Por lo tanto, el estimador es insesgado}


\miquestion 

\begin{itemize}
\item a) Si $Y_1$,$Y_2$, (...) $Y_9$  denota el contenido en onzas de las botellas que se van a observar, entonces sabemos que las $Y_i$ están distribuidas normalmente con media $\mu$ y varianza $\sigma^2$ $=$ 1 para $i $=$ 1,2, (...),9$. Por lo cual $\bar{y}$ posee una distribución muestral normal con media $$\mu_{\bar{y}} = \mu$$ y varianza 


$$\sigma^2_{\bar{y}} =  \frac{\sigma^2}{n}= 1/9 $$

\text{Tenemos que:}

%$$ P(\abs{\bar{y} (-) \mu} \leq 0.3) = $$

$$ P(-0.3 \leq (bar{y}-\mu) \leq 0.3) 
= P(\frac{-0.3}{\sigma/n} \leq \frac{(\bar{y}-\mu)}{\sigma/n} \leq \frac{0.3}{\sigma/n})
$$

\text{Como $\bar{y}$ se distribuye como una normal estándar, se deduce que: }
%$$ P(\abs{\bar{y}-\mu} \leq 0.3)  = $$

$P(\frac{-0.3}{1/9} \leq Z \leq \frac{0.3}{1/9})$

= $P(-0.9 \leq Z \leq 0.9) $ 

\text{Por simetría de una normal estándar, entonces: }

$$P(-0.9 \leq Z \leq 0.9)  = 1- 2P(Z \leq 0.9) = 1- 2(0.1841) = 0.6318
$$

En conclusión: si tomamos una muestra aleatoria de tamaño 9 de forma
iterada solo el .6318 la media muestral se encuentra a no más de 0.3 onzas de la verdadera media poblacional

\item b)
\text{Ahora buscamos: }
%$$ P(\abs{\bar{y}-\mu} \leq 0.3)  =  0.95 $$ 
=  $ P(-0.3 \leq (\bar{y}-\mu) \leq 0.3) = 0.95  $ 

\text{Entonces al estandarizar: }

$ P(\frac{-0.3}{\sigma/n} \leq \frac{(\bar{y}-\mu)}{\sigma/n} \leq \frac{0.3}{\sigma/n}) 
$ 
$ 
= P( -0.3\sqrt{n} \leq Z \leq 0.3\sqrt{n} ) = 0.95 
$

\text{Entonces al buscar en tablas: }

$$P( -1.96 \leq Z \leq 1.96 ) = 0.95 
$$ 

\text{ Para obtener el tamaño de una muestra:}

$$  -0.3\sqrt{n} = 1.96$$

\text{ o bien, lo que es equivalente:}
$
n = (\frac{1.96}{0.3})^2 = 42.68) 
$

\text{redondeando al entero próximo hacia arriba}
$ n  = 43 $ 

\end{itemize}


\miquestion 
\text{Denote a $\bar{y}$ como la media muestral de tamaño $ n = 100 $ calificaciones de una población con media $\mu = 60 $ y varianza  $\sigma^2 = 64$. }

\text{Queremos calcular }

 $ P( \bar{y} \leq 58) $. \text{Dado que sigue una distribución que se puede aproximar a una normal estándar }
 
 \text{tenemos:}
$$
 P( \bar{y} \leq 58) =  P(\frac{\bar{y}-\mu}{\sigma/\sqrt(n)} \leq \frac{58-60}{0.8}) \approx P(Z \leq -2.5) = 0.0062 
$$
 


\miquestion Como $n_1 = 6 $ y $n_2 = 10$ y las varianzas son iguales, entonces: 

$$\frac{S_1^2/\sigma_1^2 }{S_2^2/\sigma_2^2 }  = \frac{S_1^2}{S_2^2}  $$  

\text{Sabemos que tiene una distribución F con $n_1 -1 = 5 $ grados de libertad en el numerador y  $n_2 -1 = 9$ grados de libertad en el denominador. Asimismo,}

$$ P(\frac{S_1^2}{S_2^2} \leq b) =  1- P(\frac{S_1^2}{S_2^2} > b)  $$ 

\text{Por tanto, queremos determinar el número de b que delimita un área en 
la cola superior que acumula el 0.05 bajo la función
de densidad de F con 5 grados de libertad en el numerador y 
9 grados de libertad
en el denominador. Con base en tablas el valor que le corresponde es  }

$ b = 3.48$ 



\miquestion Tenemos que, si denotamos con $Y_i$ el tiempo de servicio i-ésimo cliente, entonces queremos calcular:

 
$$ 
P(\sum_{i=1}^{100} Y_i \leq 120) = P (\bar{y} \leq \frac{120}{100}) = P(\bar{y} \leq 1.20)
 $$ 

\text{Dado que el tamaño de muestra es lo suficientemente grande podemos aplicar el TLC (Teorema Central del Límite): nos dice que como $\bar{y}$ está distribuida aproximadamente como una normal con media = 1.5 y varianza 1.0/100} 

\text{Entonces: }

$$ 
P(\bar{y} \leq 1.20) = P(\frac{\bar{y}-\mu}{\sigma/\sqrt(n)} \leq \frac{1.2-1.5}{1/\sqrt(100)}) $$

$$ \approx P(Z \leq (1.2-1.5)*10) = P(Z \leq -3) = .0013
$$ 

\text{Dado que, la probabilidad de que 100 clientes puedan ser atendidos en menos de dos horas es aproximadamente 0.0013. Esta probabilidad pequeña indica que es prácticamente imposible atender a 100 clientes en menos de dos horas }


\miquestion Sea Y el número de votantes de la CDMX que están a favor de la candidata A. 
Debemos calcular $ P(\frac{y}{n} \geq 0.55) $  cuando $p$ es la probabilidad de que un votante seleccionado aleatoriamente de la CDMX esté a favor de la candidata A. 

\text{ Si consideramos los $n = 100 $ votantes de la CDMX como una muestra aleatoria de la ciudad, entonces $Y$ tiene una distribución binomial con $n = 100$ y $p = 0.5$}


\text{ Dado que el tamaño de muestra es suficientemente grande, aplicamos TCL esto implica que $\bar{x} = Y/n$ está distribuida normalmente}

\text{Entonces: }

$ p = 0.5$  y la varianza: (pq/n) = (.5)(.5) = 0.0025 

\text{por tanto}

$ P( \frac{y}{n} \geq .55 ) =$ $ P ( \frac{y/n -\mu}{\sqrt\sigma} \geq \frac{.55-.50}{0.05} $

$ \approx P( Z \geq 1) = 0.1587 $ 







\textbf{Bibliografía}
Wackerly. (2008). Estadística Matemática con Aplicaciones (7.a ed.). Cengage Learning.


\end{questions}
\end{document}
