

\documentclass{oxmathproblems}
\usepackage{blindtext}
\usepackage{hyperref}
\usepackage{geometry}
%define the page header/title info
\course{ITAM - Métodos Estadísticos para C.Pol y R.I.}
\oxfordterm{Respuestas: Repaso de Estadística 2}
\sheetnumber{1}

\sheettitle{}

\extrawidth{3cm}

\begin{document}
\begin{questions}



\miquestion 



\miquestion 

\begin{itemize}
\item a) %Si $Y_1$,$Y_2$, (...) $Y_9$  denota el contenido en onzas de las botellas que se van a observar, entonces sabemos que las $Y_i$ están distribuidas normalmente con media $\mu$ y varianza $\sigma^2$ $=$ 1 para $i $=$ 1,2, (...),9$. Por lo cual $\bar{y}$ posee una distribución muestral normal con media $$\mu_\bar{y} = \mu$$ y varianza 


$$\sigma^2_{\bar{y}} =  \frac{\sigma^2}{n}= 1/9 $$

\text{Tenemos que:}

%$$ P(\abs{\bar{y} (-) \mu} \leq 0.3) = $$

$$ P(-0.3 \leq (bar{y}-\mu) \leq 0.3) 
= P(\frac{-0.3}{\sigma/n} \leq \frac{(bar{y}-\mu)}{\sigma/n} \leq \frac{0.3}{\sigma/n})
$$

\text{Como $\bar{y}$ se distribuye como una normal estándar, se deduce que: }
%$$ P(\abs{\bar{y}-\mu} \leq 0.3)  = $$

$P(\frac{-0.3}{1/9} \leq Z \leq \frac{0.3}{1/9})$

= $P(-0.9 \leq Z \leq 0.9) $ 

\text{Por simetría de una normal estándar, entonces: }

$$P(-0.9 \leq Z \leq 0.9)  = 1- 2P(Z \leq 0.9) = 1- 2(0.1841) = 0.6318
$$

\text{En conclusión: si tomamos una muestra aleatoria de tamaño 9 de forma
iterada solo el .6318 la media muestral se encuentra a no más de 0.3 onzas de la verdadera media poblacional} 

\item b)
\text{Ahora buscamos: }
%$$ P(\abs{\bar{y}-\mu} \leq 0.3)  =  0.95 $$ 
=  $ P(-0.3 \leq (\bar{y}-\mu) \leq 0.3) = 0.95  $ 

\text{Entonces al estandarizar: }

$ P(\frac{-0.3}{\sigma/n} \leq \frac{(bar{y}-\mu)}{\sigma/n} \leq \frac{0.3}{\sigma/n}) 
$ 
$ 
= P( -0.3\sqrt{n} \leq Z \leq 0.3\sqrt{n} ) = 0.95 
$

\text{Entonces al buscar en tablas: }

$P( -1.96 \leq Z \leq 1.96 ) = 0.95 
$ 

\text{ Para obtener el tamaño de una muestra:}

$  -0.3\sqrt{n} = 1.96 #

\text{ o bien, lo que es equivalente:}
$
n = (\frac{1.96}{0.3})^2 = 42.68) 
$

\text{redondeando al entero próximo hacia arriba}
$ n  = 43 $ 

\end{itemize}


\miquestion 
Denote a $\bar{y}$ como la media muestral de tamaño $ n = 100 $ calificaciones de una población con media $\mu = 60 $ y varianza  $\sigma^2 = 64$. 
Queremos calcular 


\miquestion

\miquestion






\textbf{Bibliografía}
Wackerly. (2008). Estadística Matemática con Aplicaciones (7.a ed.). Cengage Learning.



\end{questions}
\end{document}
