

\documentclass{../oxmathproblems}
\usepackage{blindtext}
\usepackage{hyperref}
\usepackage{geometry}
%define the page header/title info
\course{ITAM - Métodos Estadísticos para C.Pol y R.I.}
\oxfordterm{Assignment 02 }
\sheetnumber{1}

\sheettitle{}

\extrawidth{3cm}

\begin{document}
\begin{questions}


\miquestion Se realizó una entrevista a 10 estudiantes acerca de sus calificaciones  esperadas y sus puntos finales en el primer parcial de matemáticas. 

\begin{tabular}{|c|c|c|}
\hline
Estudiante & Calificación esperada del examen de matemáticas & Puntos finales \\ \hline
1 & 39 & 65\\
2 & 43 & 78\\
3 & 21  & 52\\ 
4& 64 & 82\\ 
5 & 57 & 92\\ 
6 & 47 & 89\\ 
7 & 28 & 73\\
8 & 75 & 98\\ 
9 & 34 & 56 \\ 
10 & 52 & 75\\ 
\hline
\end{tabular}

\begin{itemize}
  \item Encuentre la recta de predicción de mínimos cuadrados para los datos de las calificaciones del primer parcial de matemáticas.
  \item Determine el coeficiente de correlación
   \item  ¿La correlación es significativamente distinta de cero?
  \item Determine si hay una relación lineal significativa entre las calificaciones esperadas y los puntos finales. Nota: realize una prueba de hipótesis para $\beta_0$ y $\beta_1$ % poner para bo y b1 
  \item Estime el promedio de las calificaciones para estudiantes con una  puntuación de aprovechamiento es 50, con un intervalo de confianza de 95\%
\end{itemize}
 
 
\miquestion Las estaturas y pesos de 10 jugadores atacantes de fútbol se seleccionan al azar de un equipo de estrellas de un condado. 


\begin{tabular}{|c|c|c|}
\hline
Jugador & Estatura (x) & Peso (y) \\ \hline
1 & 73 & 185\\
2 & 71& 175\\
3 & 75  & 200\\ 
4& 72 & 210\\ 
5 & 72 & 190\\ 
6 & 75 & 195\\ 
7 & 67 & 150\\
8 & 69 & 170\\ 
9 & 71 &180 \\ 
10 & 69 & 175\\ 
\hline
\end{tabular}

\begin{itemize}
  \item Determine el coeficiente de correlación para la estatura (en pulgadas) y peso (en libras). 
  \item ¿La correlación es significativamente distinta de cero?
\end{itemize}

\miquestion Se tienen los siguientes datos correspondientes:   

\begin{tabular}{|c|c|}
\hline
x & y \\ \hline
-2 & 0\\
-1 & 0\\
0 & 1\\ 
1& 1\\ 
2 & 3\\ 
\hline
\end{tabular}

\begin{itemize}
  \item a) Determine la recta de mínimos cuadrados apropiada para estos datos. 
  \item b) Determine el invervalo de confianza para E(Y) cuando x=1 con un nivel de confianza del 90\%
  \item c) Determine si los datos presentan suficiente evidencia para indicar que la pendiente difiere de 0 (con $\alpha$ = 0.05)
  \item d) Calcule un intervalo de confianza al 95\% para el parámetro  $\beta_1$
  \item e) Determine el pronostico particular para Y cuando x=2, con  (1-$\alpha$)=.90
\end{itemize}


\miquestion  En su tesis de doctorado, H. Behbahani examinó el efecto de hacer variar la proporción de agua y cemento en la resistencia del concreto después de 28 días. Para el concreto con un contenido de cemento de 200 libras por yarda cúbica. Sea (y) la resistencia y (x) la proporción de agua y cemento. 


\begin{tabular}{|c|c|}
\hline
Proporción de agua y cemento & Resistencia (100 ft/lb) \\ \hline
1.21 & 1.302\\
1.29 & 1.231\\
1.37 & 1.061\\ 
1.46 & 1.040\\ 
1.62 & 0.803\\ 
1.79 & 0.711\\ 
\hline
\end{tabular}

\begin{itemize}
  \item a) Determine la recta de mínimos cuadrados apropiada para estos datos.
  \item b) Pruebe que Ho:$\beta_1$ = 0 contra Ha:  $\beta_1$ $\neq$ 0 con $\alpha$ = 0.05. ¿Hay evidencia suficiente para decir que la resistencia tiende a disminuir con un aumento en  la proporción de agua y cemento?. 

\item c) Determine un intervalo de confianza al 90\% para la resistencia esperada del concreto cuando la proporción de agua y cemento sea de 1.5
\end{itemize}


\textbf{Bibliografía}
Wackerly. (2008). Estadística Matemática con Aplicaciones (7.a ed.). Cengage Learning.



\end{questions}
\end{document}



