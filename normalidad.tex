% Options for packages loaded elsewhere
\PassOptionsToPackage{unicode}{hyperref}
\PassOptionsToPackage{hyphens}{url}
%
\documentclass[
]{article}
\title{R Notebook}
\author{}
\date{\vspace{-2.5em}}

\usepackage{amsmath,amssymb}
\usepackage{lmodern}
\usepackage{iftex}
\ifPDFTeX
  \usepackage[T1]{fontenc}
  \usepackage[utf8]{inputenc}
  \usepackage{textcomp} % provide euro and other symbols
\else % if luatex or xetex
  \usepackage{unicode-math}
  \defaultfontfeatures{Scale=MatchLowercase}
  \defaultfontfeatures[\rmfamily]{Ligatures=TeX,Scale=1}
\fi
% Use upquote if available, for straight quotes in verbatim environments
\IfFileExists{upquote.sty}{\usepackage{upquote}}{}
\IfFileExists{microtype.sty}{% use microtype if available
  \usepackage[]{microtype}
  \UseMicrotypeSet[protrusion]{basicmath} % disable protrusion for tt fonts
}{}
\makeatletter
\@ifundefined{KOMAClassName}{% if non-KOMA class
  \IfFileExists{parskip.sty}{%
    \usepackage{parskip}
  }{% else
    \setlength{\parindent}{0pt}
    \setlength{\parskip}{6pt plus 2pt minus 1pt}}
}{% if KOMA class
  \KOMAoptions{parskip=half}}
\makeatother
\usepackage{xcolor}
\IfFileExists{xurl.sty}{\usepackage{xurl}}{} % add URL line breaks if available
\IfFileExists{bookmark.sty}{\usepackage{bookmark}}{\usepackage{hyperref}}
\hypersetup{
  pdftitle={R Notebook},
  hidelinks,
  pdfcreator={LaTeX via pandoc}}
\urlstyle{same} % disable monospaced font for URLs
\usepackage[margin=1in]{geometry}
\usepackage{color}
\usepackage{fancyvrb}
\newcommand{\VerbBar}{|}
\newcommand{\VERB}{\Verb[commandchars=\\\{\}]}
\DefineVerbatimEnvironment{Highlighting}{Verbatim}{commandchars=\\\{\}}
% Add ',fontsize=\small' for more characters per line
\usepackage{framed}
\definecolor{shadecolor}{RGB}{248,248,248}
\newenvironment{Shaded}{\begin{snugshade}}{\end{snugshade}}
\newcommand{\AlertTok}[1]{\textcolor[rgb]{0.94,0.16,0.16}{#1}}
\newcommand{\AnnotationTok}[1]{\textcolor[rgb]{0.56,0.35,0.01}{\textbf{\textit{#1}}}}
\newcommand{\AttributeTok}[1]{\textcolor[rgb]{0.77,0.63,0.00}{#1}}
\newcommand{\BaseNTok}[1]{\textcolor[rgb]{0.00,0.00,0.81}{#1}}
\newcommand{\BuiltInTok}[1]{#1}
\newcommand{\CharTok}[1]{\textcolor[rgb]{0.31,0.60,0.02}{#1}}
\newcommand{\CommentTok}[1]{\textcolor[rgb]{0.56,0.35,0.01}{\textit{#1}}}
\newcommand{\CommentVarTok}[1]{\textcolor[rgb]{0.56,0.35,0.01}{\textbf{\textit{#1}}}}
\newcommand{\ConstantTok}[1]{\textcolor[rgb]{0.00,0.00,0.00}{#1}}
\newcommand{\ControlFlowTok}[1]{\textcolor[rgb]{0.13,0.29,0.53}{\textbf{#1}}}
\newcommand{\DataTypeTok}[1]{\textcolor[rgb]{0.13,0.29,0.53}{#1}}
\newcommand{\DecValTok}[1]{\textcolor[rgb]{0.00,0.00,0.81}{#1}}
\newcommand{\DocumentationTok}[1]{\textcolor[rgb]{0.56,0.35,0.01}{\textbf{\textit{#1}}}}
\newcommand{\ErrorTok}[1]{\textcolor[rgb]{0.64,0.00,0.00}{\textbf{#1}}}
\newcommand{\ExtensionTok}[1]{#1}
\newcommand{\FloatTok}[1]{\textcolor[rgb]{0.00,0.00,0.81}{#1}}
\newcommand{\FunctionTok}[1]{\textcolor[rgb]{0.00,0.00,0.00}{#1}}
\newcommand{\ImportTok}[1]{#1}
\newcommand{\InformationTok}[1]{\textcolor[rgb]{0.56,0.35,0.01}{\textbf{\textit{#1}}}}
\newcommand{\KeywordTok}[1]{\textcolor[rgb]{0.13,0.29,0.53}{\textbf{#1}}}
\newcommand{\NormalTok}[1]{#1}
\newcommand{\OperatorTok}[1]{\textcolor[rgb]{0.81,0.36,0.00}{\textbf{#1}}}
\newcommand{\OtherTok}[1]{\textcolor[rgb]{0.56,0.35,0.01}{#1}}
\newcommand{\PreprocessorTok}[1]{\textcolor[rgb]{0.56,0.35,0.01}{\textit{#1}}}
\newcommand{\RegionMarkerTok}[1]{#1}
\newcommand{\SpecialCharTok}[1]{\textcolor[rgb]{0.00,0.00,0.00}{#1}}
\newcommand{\SpecialStringTok}[1]{\textcolor[rgb]{0.31,0.60,0.02}{#1}}
\newcommand{\StringTok}[1]{\textcolor[rgb]{0.31,0.60,0.02}{#1}}
\newcommand{\VariableTok}[1]{\textcolor[rgb]{0.00,0.00,0.00}{#1}}
\newcommand{\VerbatimStringTok}[1]{\textcolor[rgb]{0.31,0.60,0.02}{#1}}
\newcommand{\WarningTok}[1]{\textcolor[rgb]{0.56,0.35,0.01}{\textbf{\textit{#1}}}}
\usepackage{graphicx}
\makeatletter
\def\maxwidth{\ifdim\Gin@nat@width>\linewidth\linewidth\else\Gin@nat@width\fi}
\def\maxheight{\ifdim\Gin@nat@height>\textheight\textheight\else\Gin@nat@height\fi}
\makeatother
% Scale images if necessary, so that they will not overflow the page
% margins by default, and it is still possible to overwrite the defaults
% using explicit options in \includegraphics[width, height, ...]{}
\setkeys{Gin}{width=\maxwidth,height=\maxheight,keepaspectratio}
% Set default figure placement to htbp
\makeatletter
\def\fps@figure{htbp}
\makeatother
\setlength{\emergencystretch}{3em} % prevent overfull lines
\providecommand{\tightlist}{%
  \setlength{\itemsep}{0pt}\setlength{\parskip}{0pt}}
\setcounter{secnumdepth}{-\maxdimen} % remove section numbering
\ifLuaTeX
  \usepackage{selnolig}  % disable illegal ligatures
\fi

\begin{document}
\maketitle

\hypertarget{introducciuxf3n}{%
\subsubsection{1) Introducción}\label{introducciuxf3n}}

El método estándar para construir los intervalos de confianza para la
regresión lineal (y en general para conocer su distribución) recae en el
supuesto de normalidad:

\begin{itemize}
\tightlist
\item
  Los errores de la regresión están distribuidos normal
\item
  El número de observaciones es suficientemente grande, en cuyo caso, el
  estimador se distribuye normal (Teorema Central del Límite)
\end{itemize}

Si se cumple el primer supuesto, entonces el estimador de \(\beta_1\) se
distribuye normal con la media y varianza vista en clase .

Al realizar regresión lineal es importante determinar estos supuestos se
cumplen, por lo que existen distintas técnicas para lograrlo. En este
curso veremos dos tipos:

\begin{enumerate}
\def\labelenumi{\alph{enumi})}
\tightlist
\item
  Métodos Gráficos
\item
  Pruebas de Hipótesis
\end{enumerate}

\hypertarget{generaciuxf3n-de-datos}{%
\subsubsection{2) Generación de datos:}\label{generaciuxf3n-de-datos}}

Previo a comenzar con los métodos gráficos y pruebas de hipótesis se
generan datos provenientes de distintas distribuciones:

\begin{Shaded}
\begin{Highlighting}[]
\CommentTok{\# La función rnorm genera muestras aleatorias normales. Tiene como parámetros: }
\CommentTok{\# n=número de muestras, }
\CommentTok{\# mean = media de la distribucion normal }
\CommentTok{\# sd = desviacion estándar de la distribución normal}

\CommentTok{\# rt genera números aleatorios que provienen de una distribución t{-}student}

\CommentTok{\# rexp genera número aleatorios que provienen de distribución exponencial}

\FunctionTok{library}\NormalTok{(tidyverse)}
\end{Highlighting}
\end{Shaded}

\begin{verbatim}
## -- Attaching packages --------------------------------------- tidyverse 1.3.1 --
\end{verbatim}

\begin{verbatim}
## v ggplot2 3.3.6     v purrr   0.3.4
## v tibble  3.1.8     v dplyr   1.0.9
## v tidyr   1.2.0     v stringr 1.4.0
## v readr   2.1.0     v forcats 0.5.1
\end{verbatim}

\begin{verbatim}
## -- Conflicts ------------------------------------------ tidyverse_conflicts() --
## x dplyr::filter() masks stats::filter()
## x dplyr::lag()    masks stats::lag()
\end{verbatim}

\begin{Shaded}
\begin{Highlighting}[]
\FunctionTok{set.seed}\NormalTok{(}\DecValTok{532}\NormalTok{)}

\NormalTok{datos }\OtherTok{\textless{}{-}} 
  \FunctionTok{data.frame}\NormalTok{(}
  \AttributeTok{normal =} \FunctionTok{rnorm}\NormalTok{(}\AttributeTok{n =} \DecValTok{1000}\NormalTok{, }\AttributeTok{mean =} \DecValTok{0}\NormalTok{, }\AttributeTok{sd =} \DecValTok{1}\NormalTok{),}
  \AttributeTok{tstudent =} \FunctionTok{rt}\NormalTok{(}\AttributeTok{n =} \DecValTok{1000}\NormalTok{, }\AttributeTok{df =} \DecValTok{10}\NormalTok{),}
  \AttributeTok{exponencial =} \FunctionTok{rexp}\NormalTok{(}\AttributeTok{n =} \DecValTok{1000}\NormalTok{, }\AttributeTok{rate =} \FloatTok{0.5}\NormalTok{)}
\NormalTok{)}

\NormalTok{datos }\SpecialCharTok{\%\textgreater{}\%} 
  \FunctionTok{gather}\NormalTok{(variable, valor) }\SpecialCharTok{\%\textgreater{}\%} 
  \FunctionTok{ggplot}\NormalTok{(}\FunctionTok{aes}\NormalTok{(}\AttributeTok{x =}\NormalTok{valor, }\AttributeTok{color =}\NormalTok{ variable))}\SpecialCharTok{+}
  \FunctionTok{geom\_density}\NormalTok{()}\SpecialCharTok{+}
  \FunctionTok{facet\_wrap}\NormalTok{(}\SpecialCharTok{\textasciitilde{}}\NormalTok{variable)}\SpecialCharTok{+}
  \FunctionTok{theme\_minimal}\NormalTok{()}\SpecialCharTok{+}
  \FunctionTok{theme}\NormalTok{(}
    \AttributeTok{legend.position =} \StringTok{"none"}
\NormalTok{  )}\SpecialCharTok{+}
  \FunctionTok{xlim}\NormalTok{(}\SpecialCharTok{{-}}\DecValTok{5}\NormalTok{, }\DecValTok{5}\NormalTok{)}
\end{Highlighting}
\end{Shaded}

\begin{verbatim}
## Warning: Removed 76 rows containing non-finite values (stat_density).
\end{verbatim}

\includegraphics{normalidad_files/figure-latex/unnamed-chunk-1-1.pdf}

Antes de comenzar, podemos observar que la distribución Normal y la
t-student se asemejan, las dos son distribuciones simétricas, pero la
distribución t-student tiene mayor varianza:

\begin{Shaded}
\begin{Highlighting}[]
\FunctionTok{var}\NormalTok{(datos}\SpecialCharTok{$}\NormalTok{normal)}
\end{Highlighting}
\end{Shaded}

\begin{verbatim}
## [1] 1.047178
\end{verbatim}

\begin{Shaded}
\begin{Highlighting}[]
\FunctionTok{var}\NormalTok{(datos}\SpecialCharTok{$}\NormalTok{tstudent)}
\end{Highlighting}
\end{Shaded}

\begin{verbatim}
## [1] 1.310563
\end{verbatim}

A simple vista a veces se vuelve un poco complicado poder determinar
solo con la gráfica que distribución es normal y cual no.

\hypertarget{muxe9todos-gruxe1ficos}{%
\subsubsection{3) Métodos gráficos}\label{muxe9todos-gruxe1ficos}}

Los métodos gráficos tienen ciertas ventajas, entre ellas que conocemos
la distribución de los datos, los valores que toman, la frecuencia de
cada rango de valores, etc. Entre sus desventajas es que la decisión de
determinar normalidad se deja a la persona que lo esté analizando.

\hypertarget{histograma}{%
\paragraph{3.1) Histograma}\label{histograma}}

El histograma representa la información en forma de barras, donde la
superficie de cada barra implica la frecuencia de cada valor
representado. Al igual que la gráfica de la sección 2, es facil
identificar si tiene sesgo la distribución, pero se vuelve complicado
saber si tiene la varianza es equivalente a la de una distribución
normal.

\begin{Shaded}
\begin{Highlighting}[]
\NormalTok{datos }\SpecialCharTok{\%\textgreater{}\%} 
  \FunctionTok{gather}\NormalTok{(variable, valor) }\SpecialCharTok{\%\textgreater{}\%} 
  \FunctionTok{ggplot}\NormalTok{(}\FunctionTok{aes}\NormalTok{(}\AttributeTok{x =}\NormalTok{ valor, }\AttributeTok{fill =}\NormalTok{ variable))}\SpecialCharTok{+}
  \FunctionTok{geom\_histogram}\NormalTok{()}\SpecialCharTok{+}
  \FunctionTok{facet\_wrap}\NormalTok{(}\SpecialCharTok{\textasciitilde{}}\NormalTok{variable)}\SpecialCharTok{+}
  \FunctionTok{theme\_minimal}\NormalTok{()}\SpecialCharTok{+}
  \FunctionTok{theme}\NormalTok{(}\AttributeTok{legend.position =} \StringTok{"none"}\NormalTok{)}\SpecialCharTok{+}
  \FunctionTok{xlim}\NormalTok{(}\SpecialCharTok{{-}}\DecValTok{5}\NormalTok{,}\DecValTok{5}\NormalTok{)}
\end{Highlighting}
\end{Shaded}

\begin{verbatim}
## `stat_bin()` using `bins = 30`. Pick better value with `binwidth`.
\end{verbatim}

\begin{verbatim}
## Warning: Removed 76 rows containing non-finite values (stat_bin).
\end{verbatim}

\includegraphics{normalidad_files/figure-latex/unnamed-chunk-3-1.pdf}

\hypertarget{boxplot}{%
\paragraph{3.2 Boxplot}\label{boxplot}}

Los Boxplot nos permite conocer las principales estadísticas de los
datos: mínimo, primer cuartil, mediana, tercer cuartil, máximo, datos
atípicos. Con estos datos podemos determinar si la distribución de los
datos es simétrica y tambien darnos una idea de que tanto varian. Por
ejemplo, en la siguiente gráfica podemos notar:

\begin{itemize}
\tightlist
\item
  Los datos exponenciales no provienen de una distribución simétrica
\item
  Es dificil determinar si la varianza es adecuada para una normal, o
  bien tiene colas más pesadas como la t-student
\end{itemize}

\begin{Shaded}
\begin{Highlighting}[]
\NormalTok{datos }\SpecialCharTok{\%\textgreater{}\%} 
  \FunctionTok{gather}\NormalTok{(variable, valor) }\SpecialCharTok{\%\textgreater{}\%} 
  \FunctionTok{ggplot}\NormalTok{(}\FunctionTok{aes}\NormalTok{(}\AttributeTok{x =}\NormalTok{ valor, }\AttributeTok{color =}\NormalTok{ variable))}\SpecialCharTok{+}
  \FunctionTok{geom\_boxplot}\NormalTok{()}\SpecialCharTok{+}
  \FunctionTok{facet\_wrap}\NormalTok{(}\SpecialCharTok{\textasciitilde{}}\NormalTok{variable)}\SpecialCharTok{+}
  \FunctionTok{coord\_flip}\NormalTok{()}\SpecialCharTok{+}
  \FunctionTok{theme\_minimal}\NormalTok{()}\SpecialCharTok{+}
  \FunctionTok{theme}\NormalTok{(}\AttributeTok{legend.position =} \StringTok{"none"}\NormalTok{)}\SpecialCharTok{+}
  \FunctionTok{xlim}\NormalTok{(}\SpecialCharTok{{-}}\DecValTok{5}\NormalTok{,}\DecValTok{5}\NormalTok{)}
\end{Highlighting}
\end{Shaded}

\begin{verbatim}
## Warning: Removed 76 rows containing non-finite values (stat_boxplot).
\end{verbatim}

\includegraphics{normalidad_files/figure-latex/unnamed-chunk-4-1.pdf}
Por esto, el boxplot nos permite disernir si la distribución está
sesgada, pero no es sencillo determinar si la varianza corresponde a una
distribución normal.

\hypertarget{qqplot-quantile-quantile-plot}{%
\paragraph{3.3 qqplot (quantile-quantile
plot)}\label{qqplot-quantile-quantile-plot}}

La intucion detrás del qqplot es que los cuantiles de nuestros datos
(quantil empírico) deben estar ``en linea'' perfecta con los quantiles
teóricos de una distribución normal. Para facilitar en entendimiento,
veamos las siguientes gráficas:

\begin{Shaded}
\begin{Highlighting}[]
\NormalTok{datos }\SpecialCharTok{\%\textgreater{}\%} 
  \FunctionTok{gather}\NormalTok{(variable, valor) }\SpecialCharTok{\%\textgreater{}\%} 
  \FunctionTok{ggplot}\NormalTok{(}\FunctionTok{aes}\NormalTok{(}\AttributeTok{sample =}\NormalTok{ valor, }\AttributeTok{color =}\NormalTok{ variable))}\SpecialCharTok{+}
  \FunctionTok{stat\_qq}\NormalTok{() }\SpecialCharTok{+} \FunctionTok{stat\_qq\_line}\NormalTok{()}\SpecialCharTok{+}
  \FunctionTok{facet\_wrap}\NormalTok{(}\SpecialCharTok{\textasciitilde{}}\NormalTok{variable)}\SpecialCharTok{+}
  \FunctionTok{theme\_minimal}\NormalTok{()}\SpecialCharTok{+}
  \FunctionTok{theme}\NormalTok{(}\AttributeTok{legend.position =} \StringTok{"none"}\NormalTok{)}\SpecialCharTok{+}
  \FunctionTok{labs}\NormalTok{(}\AttributeTok{x =} \StringTok{"Quantiles normales teóricos"}\NormalTok{,}
       \AttributeTok{y =} \StringTok{"Quantiles normales empírico (datos)"}\NormalTok{)}
\end{Highlighting}
\end{Shaded}

\includegraphics{normalidad_files/figure-latex/unnamed-chunk-5-1.pdf} En
la gráfica aparece una línea y varios puntos. Cada uno de los puntos
corresponde a un dato. ¿Cómo se lee esta gráfica? Agarremos un punto:

\begin{Shaded}
\begin{Highlighting}[]
\NormalTok{datos }\SpecialCharTok{\%\textgreater{}\%} 
  \FunctionTok{gather}\NormalTok{(variable, valor) }\SpecialCharTok{\%\textgreater{}\%} \FunctionTok{filter}\NormalTok{(variable }\SpecialCharTok{==} \StringTok{"exponencial"}\NormalTok{) }\SpecialCharTok{\%\textgreater{}\%} 
  \FunctionTok{ggplot}\NormalTok{(}\FunctionTok{aes}\NormalTok{(}\AttributeTok{sample =}\NormalTok{ valor, }\AttributeTok{color =}\NormalTok{ variable))}\SpecialCharTok{+}
  \FunctionTok{stat\_qq}\NormalTok{() }\SpecialCharTok{+} \FunctionTok{stat\_qq\_line}\NormalTok{()}\SpecialCharTok{+}
  \FunctionTok{facet\_wrap}\NormalTok{(}\SpecialCharTok{\textasciitilde{}}\NormalTok{variable)}\SpecialCharTok{+}
  \FunctionTok{theme\_minimal}\NormalTok{()}\SpecialCharTok{+}
  \FunctionTok{geom\_hline}\NormalTok{(}\AttributeTok{yintercept =} \DecValTok{0}\NormalTok{, }\AttributeTok{color =} \StringTok{"cyan"}\NormalTok{)}\SpecialCharTok{+}
  \FunctionTok{geom\_vline}\NormalTok{(}\AttributeTok{xintercept =} \SpecialCharTok{{-}}\FloatTok{3.28}\NormalTok{, }\AttributeTok{color =} \StringTok{"cyan"}\NormalTok{)}\SpecialCharTok{+}
  \FunctionTok{theme}\NormalTok{(}\AttributeTok{legend.position =} \StringTok{"none"}\NormalTok{)}\SpecialCharTok{+}
  \FunctionTok{labs}\NormalTok{(}\AttributeTok{x =} \StringTok{"Quantiles normales teóricos"}\NormalTok{,}
       \AttributeTok{y =} \StringTok{"Quantiles normales empírico (datos)"}\NormalTok{)}
\end{Highlighting}
\end{Shaded}

\includegraphics{normalidad_files/figure-latex/unnamed-chunk-6-1.pdf}

El dato señalado es el valor empírico 0.0016. Se debe encontrar en que
posición estaría si ordenaramos los datos de menor a mayor, en este caso
es el mínimo, entonces es el dato 1 de 1000. Ahora, hay que encontrar a
que cuantil corresponde la probabilidad 1/1000 de una distribución
normal:

\begin{Shaded}
\begin{Highlighting}[]
\CommentTok{\# el primer dato de los datos exponenciales, que quantil representa }
\CommentTok{\# de datos normales}
\FunctionTok{qnorm}\NormalTok{(}\DecValTok{1}\SpecialCharTok{/}\DecValTok{1000}\NormalTok{) }\CommentTok{\# {-}3.09}
\end{Highlighting}
\end{Shaded}

\begin{verbatim}
## [1] -3.090232
\end{verbatim}

\begin{Shaded}
\begin{Highlighting}[]
\CommentTok{\# coordenada: ({-}3.09, 0.0016)}
\end{Highlighting}
\end{Shaded}

De esta manera, podemos generar la coordenada (-3.09, 0.0016), el primer
dato corresponde al cuantil de la normal y el segundo al dato empírico.
Veamos otro ejemplo:

Tomemos un valor distinto al mínimo, por ejemplo ejemplo el valor 0.48,
que es el número 225 de 1000 ordenado de menor a mayor:

\begin{Shaded}
\begin{Highlighting}[]
\NormalTok{datos }\SpecialCharTok{\%\textgreater{}\%} 
  \FunctionTok{gather}\NormalTok{(variable, valor) }\SpecialCharTok{\%\textgreater{}\%} \FunctionTok{filter}\NormalTok{(variable }\SpecialCharTok{==} \StringTok{"exponencial"}\NormalTok{) }\SpecialCharTok{\%\textgreater{}\%} 
  \FunctionTok{ggplot}\NormalTok{(}\FunctionTok{aes}\NormalTok{(}\AttributeTok{sample =}\NormalTok{ valor, }\AttributeTok{color =}\NormalTok{ variable))}\SpecialCharTok{+}
  \FunctionTok{stat\_qq}\NormalTok{() }\SpecialCharTok{+} \FunctionTok{stat\_qq\_line}\NormalTok{()}\SpecialCharTok{+}
  \FunctionTok{facet\_wrap}\NormalTok{(}\SpecialCharTok{\textasciitilde{}}\NormalTok{variable)}\SpecialCharTok{+}
  \FunctionTok{theme\_minimal}\NormalTok{()}\SpecialCharTok{+}
  \FunctionTok{geom\_hline}\NormalTok{(}\AttributeTok{yintercept =} \FloatTok{0.48}\NormalTok{)}\SpecialCharTok{+}
  \FunctionTok{geom\_vline}\NormalTok{(}\AttributeTok{xintercept =} \SpecialCharTok{{-}}\FloatTok{0.76}\NormalTok{)}\SpecialCharTok{+}
  \FunctionTok{theme}\NormalTok{(}\AttributeTok{legend.position =} \StringTok{"none"}\NormalTok{)}\SpecialCharTok{+}
  \FunctionTok{labs}\NormalTok{(}\AttributeTok{x =} \StringTok{"Quantiles normales teóricos"}\NormalTok{,}
       \AttributeTok{y =} \StringTok{"Quantiles normales empírico (datos)"}\NormalTok{)}
\end{Highlighting}
\end{Shaded}

\includegraphics{normalidad_files/figure-latex/unnamed-chunk-8-1.pdf}

\begin{Shaded}
\begin{Highlighting}[]
\FunctionTok{qnorm}\NormalTok{(}\DecValTok{225}\SpecialCharTok{/}\DecValTok{1000}\NormalTok{) }\CommentTok{\# {-}0.76}
\end{Highlighting}
\end{Shaded}

\begin{verbatim}
## [1] -0.755415
\end{verbatim}

\begin{Shaded}
\begin{Highlighting}[]
\CommentTok{\# coordenada: (0.48, {-}0.76)}
\end{Highlighting}
\end{Shaded}

Vemos que representa el cuantil 225/1000, que representa el quantil
teórico normal de -0.04.

Recapitulando, cada punto en la gráfica representa una observación y la
línea en cada facet representa como se comportarían las observaciones de
una distribución normal. Si los puntos se ajustan a la linea, quiere
decir que se asemeja a una normal.

\begin{itemize}
\item
  En el caso de datos normales (línea verde), los puntos coinciden casi
  al 100\% con la línea, lo que nos lleva a pensar que efectivamente se
  distribuyen normal
\item
  En el caso de los puntos azules, vemos que en centro se parecen los
  datos a la línea, pero en una de las colas no coinciden. ¿Qué
  implicaría?:
\end{itemize}

----- insertar imagen

Por eso, de los métodos gráficos, el qqplot es de los que más utilidad
nos generan, ya que no solo nos permite determinar si los datos tienen
sesgo 0, si no además, que la varianza es equivalente a la varianza de
una distribución normal.

\hypertarget{pruebas-de-hipuxf3tesis}{%
\subsubsection{4) Pruebas de Hipótesis}\label{pruebas-de-hipuxf3tesis}}

\hypertarget{prueba-de-hipuxf3tesis-jarque-bera}{%
\paragraph{4.1) Prueba de hipótesis
Jarque-Bera}\label{prueba-de-hipuxf3tesis-jarque-bera}}

Los métodos anteriores son gráficos, pero ¿qué sucede si queremos una
manera más estadística para llegar a una conclusión? R: aplicamos una
prueba de hipótesis

La prueba de hipótesis que aplicamos recibe el nombre de Jarque-Bera, la
cual tiene como \(H_0:\) Los datos se distribuyen normal vs \(H_1:\) Los
datos no se distribuyen normal. El estadístico lo llamaremos JB:

\[ JB = \frac{n}{6}( S_k^2 +\frac{1}{4}(K-3)^2) \] Donde en este caso
llamaremos \(S_k\) al coeficiente de asimetría y \(K\) a la curtosis.

\[ S_k = \frac{E(X-\mu)^2}{\sigma^3}\]
\[ K = \frac{E(X-\mu)^4}{(E(X-\mu)^2)^2}\]

El estadístico JB se distribuye de como una \(JB \sim \chi^2_{2}\) y
siempre rechazamos para valores altos de la distribución.

\hypertarget{cuxe1lculo-con-fuxf3rmulas}{%
\subparagraph{4.1.1) Cálculo con
fórmulas:}\label{cuxe1lculo-con-fuxf3rmulas}}

\begin{Shaded}
\begin{Highlighting}[]
\FunctionTok{library}\NormalTok{(tidyverse)}
\FunctionTok{library}\NormalTok{(moments)}
\FunctionTok{library}\NormalTok{(tseries)}
\end{Highlighting}
\end{Shaded}

\begin{verbatim}
## Registered S3 method overwritten by 'quantmod':
##   method            from
##   as.zoo.data.frame zoo
\end{verbatim}

\begin{Shaded}
\begin{Highlighting}[]
\CommentTok{\# Jarque{-}Bera }
\NormalTok{nrows }\OtherTok{=} \FunctionTok{nrow}\NormalTok{(datos)}

\CommentTok{\# normal}
\NormalTok{skew\_normal }\OtherTok{=}\NormalTok{ (}\FunctionTok{sum}\NormalTok{((datos}\SpecialCharTok{$}\NormalTok{normal}\SpecialCharTok{{-}}\FunctionTok{mean}\NormalTok{(datos}\SpecialCharTok{$}\NormalTok{normal))}\SpecialCharTok{**}\DecValTok{3}\NormalTok{)}\SpecialCharTok{/}\NormalTok{nrows) }\SpecialCharTok{/}\NormalTok{ (}\FunctionTok{var}\NormalTok{(datos}\SpecialCharTok{$}\NormalTok{normal)}\SpecialCharTok{*}\NormalTok{(nrows}\DecValTok{{-}1}\NormalTok{)}\SpecialCharTok{/}\NormalTok{nrows)}\SpecialCharTok{**}\NormalTok{(}\DecValTok{3}\SpecialCharTok{/}\DecValTok{2}\NormalTok{)}

\NormalTok{kurt\_normal }\OtherTok{=}\NormalTok{  (}\FunctionTok{sum}\NormalTok{((datos}\SpecialCharTok{$}\NormalTok{normal}\SpecialCharTok{{-}}\FunctionTok{mean}\NormalTok{(datos}\SpecialCharTok{$}\NormalTok{normal))}\SpecialCharTok{**}\DecValTok{4}\NormalTok{)}\SpecialCharTok{/}\NormalTok{nrows) }\SpecialCharTok{/}\NormalTok{ (}\FunctionTok{var}\NormalTok{(datos}\SpecialCharTok{$}\NormalTok{normal)}\SpecialCharTok{*}\NormalTok{(nrows}\DecValTok{{-}1}\NormalTok{)}\SpecialCharTok{/}\NormalTok{nrows)}\SpecialCharTok{**}\NormalTok{(}\DecValTok{4}\SpecialCharTok{/}\DecValTok{2}\NormalTok{)}

\NormalTok{JB\_normal }\OtherTok{=}\NormalTok{ nrows}\SpecialCharTok{*}\NormalTok{((skew\_normal}\SpecialCharTok{\^{}}\DecValTok{2}\NormalTok{)}\SpecialCharTok{/}\DecValTok{6} \SpecialCharTok{+}\NormalTok{ (kurt\_normal}\DecValTok{{-}3}\NormalTok{)}\SpecialCharTok{*}\NormalTok{(kurt\_normal}\DecValTok{{-}3}\NormalTok{)}\SpecialCharTok{/}\NormalTok{(}\DecValTok{24}\NormalTok{))}
\FunctionTok{pchisq}\NormalTok{(JB\_normal, }\AttributeTok{lower.tail =}\NormalTok{ F, }\AttributeTok{df =} \DecValTok{2}\NormalTok{)}
\end{Highlighting}
\end{Shaded}

\begin{verbatim}
## [1] 0.9900528
\end{verbatim}

\begin{Shaded}
\begin{Highlighting}[]
\CommentTok{\# valorp = 0.99 {-}\textgreater{} No Rechazamos H0}

\CommentTok{\# t{-}student}
\NormalTok{skew\_t }\OtherTok{=}\NormalTok{ (}\FunctionTok{sum}\NormalTok{((datos}\SpecialCharTok{$}\NormalTok{tstudent}\SpecialCharTok{{-}}\FunctionTok{mean}\NormalTok{(datos}\SpecialCharTok{$}\NormalTok{tstudent))}\SpecialCharTok{**}\DecValTok{3}\NormalTok{)}\SpecialCharTok{/}\NormalTok{nrows) }\SpecialCharTok{/}\NormalTok{ (}\FunctionTok{var}\NormalTok{(datos}\SpecialCharTok{$}\NormalTok{tstudent)}\SpecialCharTok{*}\NormalTok{(nrows}\DecValTok{{-}1}\NormalTok{)}\SpecialCharTok{/}\NormalTok{nrows)}\SpecialCharTok{**}\NormalTok{(}\DecValTok{3}\SpecialCharTok{/}\DecValTok{2}\NormalTok{)}

\NormalTok{kurt\_t }\OtherTok{=}\NormalTok{  (}\FunctionTok{sum}\NormalTok{((datos}\SpecialCharTok{$}\NormalTok{tstudent}\SpecialCharTok{{-}}\FunctionTok{mean}\NormalTok{(datos}\SpecialCharTok{$}\NormalTok{tstudent))}\SpecialCharTok{**}\DecValTok{4}\NormalTok{)}\SpecialCharTok{/}\NormalTok{nrows) }\SpecialCharTok{/}\NormalTok{ (}\FunctionTok{var}\NormalTok{(datos}\SpecialCharTok{$}\NormalTok{tstudent)}\SpecialCharTok{*}\NormalTok{(nrows}\DecValTok{{-}1}\NormalTok{)}\SpecialCharTok{/}\NormalTok{nrows)}\SpecialCharTok{**}\NormalTok{(}\DecValTok{4}\SpecialCharTok{/}\DecValTok{2}\NormalTok{)}

\NormalTok{JB\_t }\OtherTok{=}\NormalTok{ nrows}\SpecialCharTok{*}\NormalTok{((skew\_t}\SpecialCharTok{\^{}}\DecValTok{2}\NormalTok{)}\SpecialCharTok{/}\DecValTok{6} \SpecialCharTok{+}\NormalTok{ (kurt\_t}\DecValTok{{-}3}\NormalTok{)}\SpecialCharTok{*}\NormalTok{(kurt\_t}\DecValTok{{-}3}\NormalTok{)}\SpecialCharTok{/}\NormalTok{(}\DecValTok{24}\NormalTok{))}

\FunctionTok{pchisq}\NormalTok{(JB\_t, }\AttributeTok{lower.tail =}\NormalTok{ F, }\AttributeTok{df =} \DecValTok{2}\NormalTok{)}
\end{Highlighting}
\end{Shaded}

\begin{verbatim}
## [1] 0.0007907831
\end{verbatim}

\begin{Shaded}
\begin{Highlighting}[]
\CommentTok{\# valorp = 0.0007 {-}\textgreater{} Rechazamos H0}

\CommentTok{\# exponencial}
\NormalTok{skew\_exp }\OtherTok{=} \FunctionTok{sum}\NormalTok{((datos}\SpecialCharTok{$}\NormalTok{exponencial}\SpecialCharTok{{-}}\FunctionTok{mean}\NormalTok{(datos}\SpecialCharTok{$}\NormalTok{exponencial))}\SpecialCharTok{**}\DecValTok{3}\NormalTok{)}\SpecialCharTok{/}\NormalTok{nrows }\SpecialCharTok{/}\NormalTok{ (}\FunctionTok{var}\NormalTok{(datos}\SpecialCharTok{$}\NormalTok{exponencial)}\SpecialCharTok{*}\NormalTok{(nrows}\DecValTok{{-}1}\NormalTok{)}\SpecialCharTok{/}\NormalTok{nrows)}\SpecialCharTok{**}\NormalTok{(}\DecValTok{3}\SpecialCharTok{/}\DecValTok{2}\NormalTok{)}

\NormalTok{kurt\_exp }\OtherTok{=}\NormalTok{  (}\FunctionTok{sum}\NormalTok{((datos}\SpecialCharTok{$}\NormalTok{exponencial}\SpecialCharTok{{-}}\FunctionTok{mean}\NormalTok{(datos}\SpecialCharTok{$}\NormalTok{exponencial))}\SpecialCharTok{**}\DecValTok{4}\NormalTok{)}\SpecialCharTok{/}\NormalTok{nrows) }\SpecialCharTok{/}\NormalTok{ (}\FunctionTok{var}\NormalTok{(datos}\SpecialCharTok{$}\NormalTok{exponencial)}\SpecialCharTok{*}\NormalTok{(nrows}\DecValTok{{-}1}\NormalTok{)}\SpecialCharTok{/}\NormalTok{nrows)}\SpecialCharTok{**}\NormalTok{(}\DecValTok{4}\SpecialCharTok{/}\DecValTok{2}\NormalTok{)}

\NormalTok{JB\_exp }\OtherTok{=}\NormalTok{ nrows}\SpecialCharTok{*}\NormalTok{((skew\_exp}\SpecialCharTok{\^{}}\DecValTok{2}\NormalTok{)}\SpecialCharTok{/}\DecValTok{6} \SpecialCharTok{+}\NormalTok{ (kurt\_exp}\DecValTok{{-}3}\NormalTok{)}\SpecialCharTok{*}\NormalTok{(kurt\_exp}\DecValTok{{-}3}\NormalTok{)}\SpecialCharTok{/}\NormalTok{(}\DecValTok{24}\NormalTok{)) }
\FunctionTok{pchisq}\NormalTok{(JB\_exp, }\AttributeTok{lower.tail =}\NormalTok{ F, }\AttributeTok{df =} \DecValTok{2}\NormalTok{)}
\end{Highlighting}
\end{Shaded}

\begin{verbatim}
## [1] 0
\end{verbatim}

\begin{Shaded}
\begin{Highlighting}[]
\CommentTok{\# valorp = 0 {-}\textgreater{} Rechazamos H0}
\end{Highlighting}
\end{Shaded}

\hypertarget{cuxe1lculo-con-paquetes}{%
\subparagraph{4.1.2) Cálculo con
paquetes:}\label{cuxe1lculo-con-paquetes}}

\begin{Shaded}
\begin{Highlighting}[]
\CommentTok{\# formulas para sesgo y kurtosis}
\FunctionTok{skewness}\NormalTok{(datos}\SpecialCharTok{$}\NormalTok{normal)}
\end{Highlighting}
\end{Shaded}

\begin{verbatim}
## [1] 0.004523786
\end{verbatim}

\begin{Shaded}
\begin{Highlighting}[]
\FunctionTok{kurtosis}\NormalTok{(datos}\SpecialCharTok{$}\NormalTok{normal)}
\end{Highlighting}
\end{Shaded}

\begin{verbatim}
## [1] 3.01995
\end{verbatim}

Jarque bera con función

\begin{Shaded}
\begin{Highlighting}[]
\CommentTok{\# coinciden con valores de arriba}
\FunctionTok{jarque.bera.test}\NormalTok{(datos}\SpecialCharTok{$}\NormalTok{normal)}
\end{Highlighting}
\end{Shaded}

\begin{verbatim}
## 
##  Jarque Bera Test
## 
## data:  datos$normal
## X-squared = 0.019994, df = 2, p-value = 0.9901
\end{verbatim}

\begin{Shaded}
\begin{Highlighting}[]
\FunctionTok{jarque.bera.test}\NormalTok{(datos}\SpecialCharTok{$}\NormalTok{tstudent)}
\end{Highlighting}
\end{Shaded}

\begin{verbatim}
## 
##  Jarque Bera Test
## 
## data:  datos$tstudent
## X-squared = 14.285, df = 2, p-value = 0.0007908
\end{verbatim}

\begin{Shaded}
\begin{Highlighting}[]
\FunctionTok{jarque.bera.test}\NormalTok{(datos}\SpecialCharTok{$}\NormalTok{exponencial)}
\end{Highlighting}
\end{Shaded}

\begin{verbatim}
## 
##  Jarque Bera Test
## 
## data:  datos$exponencial
## X-squared = 2802.6, df = 2, p-value < 2.2e-16
\end{verbatim}

\hypertarget{ejercicio}{%
\subsubsection{5) Ejercicio:}\label{ejercicio}}

Se usarán los datos de lecturas anteriores de house\_rent. El modelo a
ajustar es: \(rent \sim size\). Es decir nos gustaría poder ajustar el
precio solamente con el tamaño de la casa.

\hypertarget{ajusta-la-regresiuxf3n-lineal-y-determina-si-la-beta_0-beta_1-y-r2-son-significativas}{%
\paragraph{\texorpdfstring{5.1) Ajusta la regresión lineal y determina
si la \(\beta_0\), \(\beta_1\) y \(R^2\) son
significativas:}{5.1) Ajusta la regresión lineal y determina si la \textbackslash beta\_0, \textbackslash beta\_1 y R\^{}2 son significativas:}}\label{ajusta-la-regresiuxf3n-lineal-y-determina-si-la-beta_0-beta_1-y-r2-son-significativas}}

\begin{Shaded}
\begin{Highlighting}[]
\FunctionTok{library}\NormalTok{(tidymodels)}
\end{Highlighting}
\end{Shaded}

\begin{verbatim}
## -- Attaching packages -------------------------------------- tidymodels 1.0.0 --
\end{verbatim}

\begin{verbatim}
## v broom        1.0.0     v rsample      1.1.0
## v dials        1.0.0     v tune         1.0.0
## v infer        1.0.3     v workflows    1.0.0
## v modeldata    1.0.0     v workflowsets 1.0.0
## v parsnip      1.0.1     v yardstick    1.0.0
## v recipes      1.0.1
\end{verbatim}

\begin{verbatim}
## -- Conflicts ----------------------------------------- tidymodels_conflicts() --
## x scales::discard() masks purrr::discard()
## x dplyr::filter()   masks stats::filter()
## x recipes::fixed()  masks stringr::fixed()
## x dplyr::lag()      masks stats::lag()
## x yardstick::spec() masks readr::spec()
## x recipes::step()   masks stats::step()
## * Search for functions across packages at https://www.tidymodels.org/find/
\end{verbatim}

\begin{Shaded}
\begin{Highlighting}[]
\NormalTok{house\_rent }\OtherTok{\textless{}{-}} 
  \FunctionTok{read\_csv}\NormalTok{(}\StringTok{"notas\_r/house\_rent.csv"}\NormalTok{) }\SpecialCharTok{\%\textgreater{}\%} 
  \FunctionTok{select}\NormalTok{(Rent, Size) }\SpecialCharTok{\%\textgreater{}\%} 
  \FunctionTok{set\_names}\NormalTok{(}\FunctionTok{c}\NormalTok{(}\StringTok{"rent"}\NormalTok{, }\StringTok{"size"}\NormalTok{))}
\end{Highlighting}
\end{Shaded}

\begin{verbatim}
## Rows: 4746 Columns: 12
\end{verbatim}

\begin{verbatim}
## -- Column specification --------------------------------------------------------
## Delimiter: ","
## chr  (7): Floor, Area Type, Area Locality, City, Furnishing Status, Tenant P...
## dbl  (4): BHK, Rent, Size, Bathroom
## date (1): Posted On
## 
## i Use `spec()` to retrieve the full column specification for this data.
## i Specify the column types or set `show_col_types = FALSE` to quiet this message.
\end{verbatim}

\begin{Shaded}
\begin{Highlighting}[]
\NormalTok{lm\_fit }\OtherTok{\textless{}{-}} 
  \FunctionTok{linear\_reg}\NormalTok{() }\SpecialCharTok{\%\textgreater{}\%} 
  \FunctionTok{fit}\NormalTok{(rent}\SpecialCharTok{\textasciitilde{}}\NormalTok{size, }\AttributeTok{data =}\NormalTok{ house\_rent)}

\FunctionTok{tidy}\NormalTok{(lm\_fit)}
\end{Highlighting}
\end{Shaded}

\begin{verbatim}
## # A tibble: 2 x 5
##   term        estimate std.error statistic   p.value
##   <chr>          <dbl>     <dbl>     <dbl>     <dbl>
## 1 (Intercept) -14282.    1883.       -7.58 4.01e- 14
## 2 size            50.9      1.63     31.3  1.69e-195
\end{verbatim}

\begin{Shaded}
\begin{Highlighting}[]
\FunctionTok{glance}\NormalTok{(lm\_fit)}
\end{Highlighting}
\end{Shaded}

\begin{verbatim}
## # A tibble: 1 x 12
##   r.squared adj.r~1  sigma stati~2   p.value    df  logLik    AIC    BIC devia~3
##       <dbl>   <dbl>  <dbl>   <dbl>     <dbl> <dbl>   <dbl>  <dbl>  <dbl>   <dbl>
## 1     0.171   0.171 71122.    979. 1.69e-195     1 -59756. 1.20e5 1.20e5 2.40e13
## # ... with 2 more variables: df.residual <int>, nobs <int>, and abbreviated
## #   variable names 1: adj.r.squared, 2: statistic, 3: deviance
\end{verbatim}

Aunque \(R^2 = 0.17\), vemos que el valorp es cercano a 0, por lo que
podemos rechazar \(H_0: P^2= 0\). Entonces si es significativa. La misma
conclusión se puede derivar de \(\beta_0\) y \(\beta_1\).

Los residuos los podemos calcular:

\begin{Shaded}
\begin{Highlighting}[]
\NormalTok{yhat }\OtherTok{\textless{}{-}} \FunctionTok{predict}\NormalTok{(lm\_fit, }\AttributeTok{new\_data =}\NormalTok{ house\_rent }\SpecialCharTok{\%\textgreater{}\%} \FunctionTok{select}\NormalTok{(size))}
\NormalTok{residuo }\OtherTok{\textless{}{-}}\NormalTok{ house\_rent}\SpecialCharTok{$}\NormalTok{rent}\SpecialCharTok{{-}}\NormalTok{yhat}
\end{Highlighting}
\end{Shaded}

\hypertarget{realiza-un-histograma-para-mostrar-la-distribuciuxf3n-de-los-errores-de-la-regresiuxf3n}{%
\paragraph{5.2) Realiza un histograma para mostrar la distribución de
los errores de la
regresión}\label{realiza-un-histograma-para-mostrar-la-distribuciuxf3n-de-los-errores-de-la-regresiuxf3n}}

\begin{Shaded}
\begin{Highlighting}[]
\NormalTok{residuo }\SpecialCharTok{\%\textgreater{}\%} 
  \FunctionTok{ggplot}\NormalTok{(}\FunctionTok{aes}\NormalTok{(}\AttributeTok{x =}\NormalTok{ .pred)) }\SpecialCharTok{+}
  \FunctionTok{geom\_histogram}\NormalTok{(}\AttributeTok{fill =} \StringTok{"blue"}\NormalTok{)}\SpecialCharTok{+}
  \FunctionTok{theme\_minimal}\NormalTok{()}\SpecialCharTok{+}
  \FunctionTok{theme}\NormalTok{(}\AttributeTok{legend.position =} \StringTok{"none"}\NormalTok{)}\SpecialCharTok{+}
  \FunctionTok{xlim}\NormalTok{(}\SpecialCharTok{{-}}\DecValTok{100000}\NormalTok{,}\DecValTok{100000}\NormalTok{)}
\end{Highlighting}
\end{Shaded}

\begin{verbatim}
## `stat_bin()` using `bins = 30`. Pick better value with `binwidth`.
\end{verbatim}

\begin{verbatim}
## Warning: Removed 173 rows containing non-finite values (stat_bin).
\end{verbatim}

\includegraphics{normalidad_files/figure-latex/unnamed-chunk-15-1.pdf}
Se puede observar que hay un ligero sesgo a la derecha, pero realmente
con la gráfica no podemos concluir

\hypertarget{realiza-un-boxplot-para-mostrar-la-distribuciuxf3n-de-los-errores-de-la-regresiuxf3n}{%
\paragraph{5.2) Realiza un boxplot para mostrar la distribución de los
errores de la
regresión}\label{realiza-un-boxplot-para-mostrar-la-distribuciuxf3n-de-los-errores-de-la-regresiuxf3n}}

\begin{Shaded}
\begin{Highlighting}[]
\NormalTok{residuo }\SpecialCharTok{\%\textgreater{}\%} 
  \FunctionTok{ggplot}\NormalTok{(}\FunctionTok{aes}\NormalTok{(}\AttributeTok{x =}\NormalTok{ .pred))}\SpecialCharTok{+}
  \FunctionTok{geom\_boxplot}\NormalTok{()}\SpecialCharTok{+}
  \FunctionTok{coord\_flip}\NormalTok{()}\SpecialCharTok{+}
  \FunctionTok{theme\_minimal}\NormalTok{()}\SpecialCharTok{+}
  \FunctionTok{theme}\NormalTok{(}\AttributeTok{legend.position =} \StringTok{"none"}\NormalTok{)}\SpecialCharTok{+}
  \FunctionTok{xlim}\NormalTok{(}\SpecialCharTok{{-}}\DecValTok{100000}\NormalTok{,}\DecValTok{100000}\NormalTok{)}
\end{Highlighting}
\end{Shaded}

\begin{verbatim}
## Warning: Removed 173 rows containing non-finite values (stat_boxplot).
\end{verbatim}

\includegraphics{normalidad_files/figure-latex/unnamed-chunk-16-1.pdf}

En esta gráfica es un poco más complicado determinar si hay un sesgo o
incluso si corresponde a una varianza de una distribución normal

\hypertarget{realiza-un-qqplot-para-mostrar-la-distribuciuxf3n-de-los-errores-de-la-regresiuxf3n}{%
\paragraph{5.3) Realiza un qqplot para mostrar la distribución de los
errores de la
regresión}\label{realiza-un-qqplot-para-mostrar-la-distribuciuxf3n-de-los-errores-de-la-regresiuxf3n}}

\begin{Shaded}
\begin{Highlighting}[]
\NormalTok{residuo }\SpecialCharTok{\%\textgreater{}\%} 
  \FunctionTok{ggplot}\NormalTok{(}\FunctionTok{aes}\NormalTok{(}\AttributeTok{sample =}\NormalTok{ .pred))}\SpecialCharTok{+}
  \FunctionTok{stat\_qq}\NormalTok{() }\SpecialCharTok{+} \FunctionTok{stat\_qq\_line}\NormalTok{()}\SpecialCharTok{+}
  \FunctionTok{theme\_minimal}\NormalTok{()}\SpecialCharTok{+}
  \FunctionTok{theme}\NormalTok{(}\AttributeTok{legend.position =} \StringTok{"none"}\NormalTok{)}\SpecialCharTok{+}
  \FunctionTok{labs}\NormalTok{(}\AttributeTok{x =} \StringTok{"Quantiles normales teóricos"}\NormalTok{,}
       \AttributeTok{y =} \StringTok{"Quantiles normales empírico (datos)"}\NormalTok{)}
\end{Highlighting}
\end{Shaded}

\includegraphics{normalidad_files/figure-latex/unnamed-chunk-17-1.pdf}
Con la gráfica podemos observar que tiene colas más pesadas que una
distribución normal, por lo que probablemente el supuesto de normalidad
no se cumpla.

\hypertarget{realiza-una-prueba-jarque-bera-para-mostrar-la-distribuciuxf3n-de-los-errores-de-la-regresiuxf3n}{%
\paragraph{5.3) Realiza una prueba Jarque-Bera para mostrar la
distribución de los errores de la
regresión}\label{realiza-una-prueba-jarque-bera-para-mostrar-la-distribuciuxf3n-de-los-errores-de-la-regresiuxf3n}}

\begin{Shaded}
\begin{Highlighting}[]
\FunctionTok{jarque.bera.test}\NormalTok{(residuo}\SpecialCharTok{$}\NormalTok{.pred)}
\end{Highlighting}
\end{Shaded}

\begin{verbatim}
## 
##  Jarque Bera Test
## 
## data:  residuo$.pred
## X-squared = 240308034, df = 2, p-value < 2.2e-16
\end{verbatim}

Por lo tanto rechazamos que los datos sean normales.

\hypertarget{comprueba-que-coverrores-varerror}{%
\paragraph{5.3) Comprueba que cov(errores),
var(error),}\label{comprueba-que-coverrores-varerror}}

\end{document}
