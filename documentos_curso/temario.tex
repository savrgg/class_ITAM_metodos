
%Example of use of oxmathproblems latex class for problem sheets
%(un)comment this line to enable/disable output of any solutions in the file
%\printanswers
\documentclass{../oxmathproblems}
\usepackage{blindtext}
\usepackage{hyperref}
\usepackage{geometry}
%define the page header/title info
\course{ITAM - Métodos Estadísticos para C.Pol y R.I.}
\oxfordterm{Temario}
\sheetnumber{1}

\sheettitle{}

\extrawidth{2cm}

\begin{document}

\begin{questions}

\miquestion \textbf{Preeliminares de Conceptos Estadísticos y Fuentes de datos}

\begin{itemize}

\item 1.1) Conceptos básicos de Probabilidad y Estadística

\item 1.2) Tipos de variables y sus escalas de medición 

\item 1.3) Tipos de datos observacionales: transversal, longitudinal, series de tiempo

\item 1.4) Fuentes tradicionales de datos y representación en base de datos

\item 1.5) Fuentes actuales de consumo de información (DBs, APIs, etc) 

\end{itemize}

\miquestion \textbf{Modelo de Regresión Lineal Simple}

\begin{itemize}

\item 2.1) Concepto e interpretaciones de regresión 
\item 2.2) Relaciones estadísticas y relaciones deterministas
\item 2.3) Método de mínimos cuadrados ordinarios (MCO) y estimadores de mínimos cuadrados
\item 2.4) Propiedades de los estimadores minimos cuadrados (MCO)
\item 2.5) Bondad de ajuste: $r^2$, ANOVA
\item 2.5) Varianza, covarianza de los estimadores minimos cuadrados
\item 2.6) Intervalos de confianza para coeficientes de regresión
\item 2.7) Verificación de hipótesis 

\end{itemize}

\miquestion \textbf{Modelo de Regresión Lineal Simple}


\end{questions}
\textbf{Bibliografía}
Mendenhall, W. (2006). Introducción a la probabilidad y Estadística (Vol. 13). Cengage Learning.
Aguirre, V. A. B. A. (2006). Fundamentos de Probabilidad y Estadística (2 ed.). Jit Press.

\end{document}
