
%Example of use of oxmathproblems latex class for problem sheets
%(un)comment this line to enable/disable output of any solutions in the file
%\printanswers
\documentclass{../oxmathproblems}
\usepackage{blindtext}
\usepackage{hyperref}
\usepackage{geometry}
%define the page header/title info
\course{ITAM - Métodos Estadísticos para C.Pol y R.I.}
\oxfordterm{Temario}
\sheetnumber{1}

\sheettitle{}

\extrawidth{2cm}

\begin{document}


Examenes Parciales: 50\%

Tareas: 10\% 

Proyecto: 10\%

Examen Final: 30\%

\begin{questions}

\miquestion \textbf{Repaso de estadistica 1 y 2}

\begin{itemize}
\item 1.1) Repaso de probabilidad
\item 1.2) Repaso de variables aleatorias y distribuciones de probabilidad
\item 1.3) Repaso de TCL y distribuciones muestrales
\item 1.4) Repaso de Propiedades de estimadores y estimación puntual
\item 1.5) Repaso de Estimación por intervalos
\item 1.6) Repaso de Pruebas de hipótesis paramétricas
\end{itemize}

\miquestion \textbf{Modelo de Regresión Lineal Simple (RLS)}

\begin{itemize}
\item 2.1) Concepto e interpretaciones de regresión 
\item 2.2) Relaciones estadísticas y relaciones deterministas
\item 2.3) Método de mínimos cuadrados ordinarios (MCO) y estimadores de mínimos cuadrados
\item 2.4) Propiedades y Supuestos para los estimadores minimos cuadrados (MCO)
\item 2.5) Esperanza, Varianza y Covarianza de los estimadores minimos cuadrados
\item 2.6) Pruebas de Hipótesis e Intervalos de Confianza 
\item 2.7) $r^2$, ANOVA
\item 2.8) Teorema de Gauss-Markov
\item 2.9) Verificación de supuesto de normalidad
\subitem 2.9.1) Métodos gráficos: qqplot, boxplot, histograma
\subitem 2.9.2) Prueba de hipótesis de Jarque-Bera
\item 2.10) Caso variable binaria
\end{itemize}
*Examen Parcial: 21 de Septiembre 2022

\miquestion \textbf{Transformaciones y formas funcionales}

\begin{itemize}
\item 3.1) Elasticidad Constante: Log-Log
\item 3.2) Rendimientos Decrecientes: Lin-Log
\item 3.3) Crecimiento Exponencial: Log-Lin
\item 3.4) Modelo Recíproco
\item 3.5) Transformaciones Box-Cox
\end{itemize}

\miquestion \textbf{Regresión lineal múltiple (RLM)}
\begin{itemize}
\item 4.1) Problema de mínimos cuadrados
\item 4.2) Esperanza, Varianza y Covarianza de los estimadores minimos cuadrados
\item 4.3) Pruebas de Hipótesis e Intervalos de Confianza 
\item 4.4) $r^2$, ANOVA
\item 4.5) Variables Categóricas
\item 4.6) Importancia relativa de regresores
\end{itemize}
*Examen Parcial: 14 de Noviembre 2022

\miquestion \textbf{Pronósticos series de tiempo}

\begin{itemize}
\item 5.1) Introducción series de tiempo
\item 5.2) Descomposición aditiva de una serie de tiempo
\item 5.3) Modelos de regresión para series de tiempo
\item 5.4) Modelos de descomposición para series de tiempo
\item 5.5) Modelos de suavizamiento exponencial
\item 5.6) Modelos ARIMA
\item 5.7) Métricas de error: MAE, MSE, RMSE, MAPE, MASE.
\end{itemize}

\miquestion \textbf{Verificación y corrección de supuestos}
**Contingente al tiempo disponible**
\begin{itemize}
\item 6.1) Heterocedasticidad
\item 6.2) Autocorrelación
\item 6.3) Colinealidad
\item 6.4) No normalidad
\end{itemize}

\end{questions}

\textbf{Bibliografía}

\begin{itemize}
\item  Gujarati, D. N., Porter, D. C. (2011). Econometria básica. ed. Porto Alegre: AMGH.
\item Draper, Norman R., and Harry Smith. Applied regression analysis. Vol. 326. John Wiley and Sons, 1998.
\item Hyndman, Rob J., and George Athanasopoulos. Forecasting: principles and practice. OTexts, 2018.
\item Wackerly, Dennis, William Mendenhall, and Richard L. Scheaffer. Mathematical statistics with applications. Cengage Learning, 2014.
\item Gammel, John L., and Lucy Maud Montgomery. LM Montgomery and Canadian culture. University of Toronto Press, 1999.
\end{itemize}



\end{document}
