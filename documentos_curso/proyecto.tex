
%Example of use of oxmathproblems latex class for problem sheets
%(un)comment this line to enable/disable output of any solutions in the file
%\printanswers
\documentclass{../oxmathproblems}
\usepackage{blindtext}
\usepackage{hyperref}
\usepackage{geometry}
%define the page header/title info
\course{ITAM - Métodos Estadísticos para C.Pol y R.I.}
\oxfordterm{Proyecto Otoño 2022}
\sheetnumber{1}

\newcommand{\SubItem}[1]{
    {\setlength\itemindent{15pt} \item[-] #1}
}


\sheettitle{}

\extrawidth{2cm}

\begin{document}

\textbf{Objetivo:} Realizar regresiones lineales basadas en el conjunto de datos \textit{House Prices}, que contiene datos que nos permiten estimar el precio de una casa basado en sus características.

\qquad

\textbf{Descripción:} Por cada integrante del proyecto se deberá entregar un modelo lineal que busque explicar el precio de las casas. Equipos de mínimo 2 integrantes y máximo 4.

\qquad

\textbf{Entregables:}

\begin{itemize}

\item \textbf{(60pts) Documento docx o pdf}: 
\SubItem (20pts) Entre 2 y 3 hojas de introducción donde se realice un Exploratory Data Analysis (EDA) que nos permita identificar variables importantes.
\SubItem (30pts) Entre 1 y 2 hojas explicativas por modelo. Debe explicar todos los componentes vistos de Regresión Lineal Múltiple: $\hat{\beta_i}$, IC de $\beta_i$, Pruebas de hipótesis de $\beta_i$, Matriz de Varianzas-Covarianzas, IC para predicción media, para predicción a futuro, $\hat{\sigma^2}$, etc. Es importante que las explicaciones estén en contexto del problema.
\SubItem (10pts) 1 hoja de conclusión donde se comparen los modelos presentados y se concluya cual es el mejor modelo. Incluir su usuario dentro de plataforma Kaggle

El reporte escrito no debe contener código, solamente los resultados (graficas, tablas comparativas, etc) 

\item \textbf{(20 pts) Documento Rmd  y PDF:} Se debe incluir el Notebook (Rmd) con sus secciones que correspondan al reporte escrito

\item \textbf{(20 pts) Puntuación en plataforma Kaggle}

\item \textbf{Opcionales}: 
\SubItem 5 pts al mejor nombre de equipo
\SubItem 10 pts a la mejor puntuación de Kaggle
\SubItem 5 pts a la segunda mejor puntuación de Kaggle

\end{itemize}

Nota: Todo el proyecto debe ser congruente en todo su contenido; por ejemplo, los resultados del modelo de regresión deben estar de acorde a lo concluido del EDA.

\newpage

\textbf{Intrucciones:}

1. Acceda al proyecto en la plataforma de \href{https://www.kaggle.com/t/d0de518d6d7a41b8b367fa01bf92ce64}{Kaggle} y registre a su equipo e integrantes. 

\qquad 

2. Descargue los datos, son 4 archivos:
\begin{itemize}
\item train.csv: estos datos contienen todas las columnas, incluyendo la columna a predecir
\item test.csv: estos datos contienen todas las columnas, excepto la columna a predecir. De estos datos se realizará la estimación del precio y se subirá a la plataforma
\item data\_description.txt: descripción de las columnas 
\item sample\_submission.csv: contiene el formato en que deben subir los datos (solo se suben esas dos columnas)
\end{itemize}

\qquad

3. Desarrolle los modelos en R y elija el mejor

\qquad

4. Realize submit de las predicciones en el formato adecuado





\end{document}
